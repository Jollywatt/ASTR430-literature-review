\chapter{Introduction}

\textsc{The Physics of Nature} at the smallest and largest scales underwent a spectacular revolution in the twentieth century, but for two decades seems to have stagnated, troubled by a few especially stubborn mysteries.
The last century bore, in parallel, the standard model of particle physics and the general relativistic theory of gravity.
The essential incompatibility between the two calls for a major breakthrough which still remains out of sight.
Aside from this foundational issue, key unresolved problems include the apparent need for dark matter in the standard model of cosmology, and the apparently unnatural fine-tuning of the standard model of particle physics.
One such issue of fine-tuning is named the strong \CP problem, pertaining to the unexplained time-reversal symmetry in quantum chromodynamics (QCD), and the concomitant non-observation of the neutron’s electric dipole moment.
One such resolution to this problem is named the Peccei--Quinn axion solution.

The axion solution predicts the existence of a light, weakly interacting particle which naturally restores time-reversal symmetry in QCD.
Undoubtedly, extensions to the standard model are more appealing when they resolve many existing issues with the same underpinning theoretical supposition.
% replete, diverse
In this sense, the axion is an alluring solution to the strong \CP problem, potentially serving as a dark matter candidate, providing a mechanism for cosmic inflation or even explaining baryogenesis \cite{landscape_2020}.
The shortcoming is that it has never admitted any sign of existence, despite extensive searches.
Nevertheless, ``physics thrives on crisis'' as Weinberg pit it \cite{physics-thrives-on-crisis}, and in the last two decades the axion has regained popularity.\footnotemark\ 
The axion has diverse consequences for particle physics, cosmology and astrophysics, and should it ever turn out an accurate reflection of nature, would certainly reshape more than one area of physics.

\footnotetext{
	The \href{https://inspirehep.net/literature?q=axion}{publication count on Inspire-HEP} for the term ``axion'' has been exponentially growing since the new millennium, doubling every 6.5 years.%, with correlation $R = \num{0.92}$.
}

The focus of this pedagogical review is half theoretical and half phenomenological.
The first half is an overview of the theory underlying quantum chromodynamics, the strong \CP problem and its axion solution.
The overview begins assuming minimal familiarity with the standard model, giving a high-level description of the mathematical structures involved, and ends with a basic exposition of axion phenomenology.
The second half is devoted to the implications of axions in cosmology and astrophysics, and the limits and constraints on axion models derived from them.
