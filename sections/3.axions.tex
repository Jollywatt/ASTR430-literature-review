\chapter{Axions in Cosmology}

\begin{figure}[t!]
\begin{aside}[Terminology from cosmology]
\begin{itemize}[leftmargin=0.75em]\setlength\itemsep{0.25ex}
	\item \emph{thermalisation}
---	the process of a particle species reaching thermal equilibrium (i.e., uniform energy and abundance) over cosmological scales, mitigated by self-interactions or processes involving other matter species which diffuse energy.
	\item \emph{freeze-out}
---	the point beyond which the rates of thermalising processes become negligible due to a sufficiently large rate of cosmic expansion, resulting in persistent non-equilibrium distributions of a particle species.
Since the universe cools as it expands, freeze-out may be viewed as the change of phase of a species from a gas to a cooler condensate.
\end{itemize}
\end{aside}
\end{figure}

With the advent of precision cosmology, particle physicists can use the entire universe as a laboratory for ever more sensitive experiments.

Basicaly:
\begin{itemize}
	\item Model universe as homogeneous and isotropic (FLRW)
	\item State of universe is defined by the abundances of each species.
\end{itemize}

\begin{align}
	Γ_{a\to γγ} \approx \SI{1.1e-24}{\per\second}\qty(\frac{m_a}{\si{\eV}})^5
\end{align}

