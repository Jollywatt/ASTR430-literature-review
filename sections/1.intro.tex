\chapter{QCD and the Strong \CP\ Problem}

\textsc{The Standard Model} of particle physics is a quantum field theory which describes all known fundamental particles and interactions to high precision---with notable exceptions including:
the absence of gravity;
dark matter;
and the experimentally incorrect prediction that neutrinos are massless.

% With its emergence in the \note{buzz of phenomenological particle physics in the early 1960s}.
The standard model was born amid the explosion of phenomenological particle physics in the early 1960s, where the catalogue of known particles was growing rapidly as new hadrons were discovered.
The introduction of the quark model explained the properties of numerous hadrons in terms of six kinds of quark, and, together with the well-established quantum theory of electromagnetism, became the first version approximation to the standard model \cite{Weinberg_2004-history}.

The modern formulation of the standard model is as a Lorentz-invariant \emph{quantum field theory} whose \emph{gauge group} is the compact connected Lie group
\begin{align}
	\qty\big(\mathrm{SU}(3)\oplus\mathrm{SU}(2)\oplus\mathrm{U}(1))\big/\mathds{Z}_6
	\label{eqn:SM-gauge-group}
.\end{align}
A quantum field theory is a quantised \emph{gauge field theory}, which is a field theory whose Lagrangian is symmetric under the action of the gauge group (elaborated upon in §\,\ref{sec:overview-of-gauge-theory}).
Each term in the gauge group of the standard model \eqref{eqn:SM-gauge-group} corresponds to a fundamental force of nature.
% The meaning and significance of the gauge group is outlined in the next section.
The factor $\mathrm{SU}(2)\oplus\mathrm{U}(1)$ corresponds to the electroweak force of quantum electrodynamics (QED); and the factor $\mathrm{SU}(3)$ is the gauge group of quantum chromodynamics (QCD), the sector of the standard model which describes the interactions of colour-charged quarks and gluons via the strong force.

Quantum chromodynamics was formulated quite some time after quantum electrodynamics was understood, which is perhaps a reflection that QCD is more intricate than QED.
Unlike QED, quantum chromodynamics exhibits \emph{asymptotic freedom}, which results in quark confinement.
This contributed to the confusion of particle physicists in the 1960s---quarks could never be observed in isolation.
Another subtlety of QCD is that its asymptotic freedom leads to \emph{increasing} effective strength of the strong force with increasing distance between colour-charged quarks, meaning that QCD is non-perturbative.
However, quantum chromodynamics remains a highly successful and beautiful theory.
% It is successful in the sense that offers precise predictions
% Setting aside technical problems of confinement and non-perturbative dynamics, QCD exhibits in particular a 
% Aside from the technically difficult problems concerning confinement and the non-perturbative regime
Aside from the technical problems of analysis in the non-perturbative regime, QCD theory also exhibits a mystery of purely theoretical concern: the \emph{strong \CP\ problem}.
% This is a problem of fine-tuning and is a smoking-gun for physics beyond the standard model.
This problem is of the fine-tuning variety, but it lacks even an anthropic solution and is good reason to pursue physics beyond the standard model.


The significance and role of the gauge group is outlined in the next section where a modern overview of gauge field theory
% ---the formalism in which the Standard Model and QCD is expressed---
is presented.
Then, QCD is defined and the origin of the so-called \emph{strong \CP\ problem} is highlighted.


\note{Justify focus on classical gauge theory}

\section{Overview of Classical Gauge Theory}
\label{sec:overview-of-gauge-theory}

Gauge theories take place on a spacetime manifold and consist of two mathematically distinct dynamical entities: a \emph{matter field} and a \emph{gauge field}.
A classical gauge theory is completely specified by the prescription of three ingredients: the base manifold; the vector space in which the matter field takes its values; and a \emph{gauge group} equipped with an action on the matter field.
The dynamical gauge field arises naturally as a consequence of the matter field's symmetry under the action of the gauge group.

The underlying premise of gauge theory is that there may be redundancy in the mathematical description of a matter field at each point in spacetime.
(For example, the complex phase of a wavefunction.)
This redundancy results in mathematical degrees of freedom of the matter fields which are non-physical, called \emph{local gauge freedoms}.
(Keeping with our example: local rotations of phase.)
Crucially, local gauge freedoms introduce ambiguity in the notion of the physical rate of change of matter fields about a point, because of the point-wise independence of its gauge freedom.
In other words, there is no preferred directional derivative of a matter field with local gauge freedom, until the choice of a \emph{connection} is made.
The triumph of gauge theory is that, by introducing the gauge field to act as a connection, this ambiguity is promoted to a separate set of physical degrees of freedom.
The implications of this are twofold:
Firstly, the gauge field yields a choice of derivative (namely, the covariant derivative with respect to the gauge field), allowing the inclusion of well-defined derivatives of matter fields in the theory's equations of motion.
Secondly, the gauge field has a dynamical role in the theory, describing a new kind of field: force fields (and, after quantisation, the force carrier bosons).



\begin{figure}
\begin{aside}[Notations used in this chapter]
	\begin{tabular}{cl}
		% $\hookrightarrow$; $\twoheadrightarrow$; $\leftrightarrow$
		$\cal{A, B, ...}$
	&	objects with manifold structure
	\\	$\fibrebundle{}{;\,}{};\,\leftrightarrow$
	&	injection; surjection; bijection
	\\	$\fibrebundle[\pi]{F}{\cal E}{\cal M}$
	&	fibre bundle $\cal E$ over base space $\cal M$ with fibre $F$ and projection $\pi : \cal E \twoheadrightarrow \cal M$
	\\	$\T_p\cal M$; $\T\cal M$
	&	tangent space at $p \in \cal M$; tangent bundle of $\cal M$
	\\	$\T^r_s\cal M$
	&	type $\smqty(r\\s)$ tensors over $\cal M$, equal to
		\begin{math}
			\qty(\bigotimes_{i=1}^r\T\cal M) \otimes \qty(\bigotimes_{i=1}^s\T^*\cal M)
		\end{math}
	\\[.7ex]	$\TT\cal M$
	&	tensor bundle of $\cal M$, equal to
		\begin{math}
			% \bigoplus_{r,s\in\mathds N}\T^r_s\cal M
			\bigoplus_{r,s=0}^\infty\T^r_s\cal M
		\end{math}
	\\	$\ts A$
		% & Spacetime tensor; an element of $\T\cal M\otimes\cdots\otimes\T\cal M\otimes\T^*\cal M\otimes\cdots\otimes\T^*\cal M$
	&	spacetime tensor field; i.e., a section of $\TT\cal M$
	\\	$\bm A$
	&	vector field of some abstract vector space not in $\TT\cal M$
	\\	$\tsbm A$
	&	spacetime tensor field with values in some other abstract vector space
	\\	$\wedge^pV$
	&	$p$th exterior power of the vector space $V$, consisting of $p$-forms
	\\	$\secs(\cal E)$
	&	smooth sections $\cal M \to F$ of fibre bundle $\fibrebundle{F}{\cal E}{\cal M}$
	\\	$\forms^p(\cal M)$
	&	space of $p$-forms on $\cal M$, equal to $\secs(\wedge^p \T^*\cal M)$
	\\	$\forms^p(\cal M, V)$
	&	space of $V$-valued $p$-forms, equal to $\secs(\cal E\otimes\wedge^p \T^*\cal M)$ where $\fibrebundle{V}{\cal E}{\cal M}$
	% \\	$\frak g$
	% &	Lie algebra of group $G$
	% \\	$\dd$
	% &	exterior derivative
	% \\	$\nabla_{\tsbm A}$
	% &	covariant derivative with respect to $\tsbm A$
	% \\	$\dd_A$
	% &	covariant exterior derivative with respect to $\tsbm A$
	\end{tabular}
\end{aside}	
\end{figure}

In modern formalism, the matter field $\bm\psi \in \secs(\cal E)$ is a section of a vector bundle $\fibrebundle{V}{\cal E}{\cal M}$, so that at any point $p \in \cal M$ in spacetime the field assigns a vector, $\bm\psi|_p \cong V$.%
\footnote{
	The reason many objects in gauge theory (such as fields $\bm\psi$) are defined as sections of fibre bundles (and not simply as smooth maps $\bm\psi : \cal M \to V$) is because fibre bundles are themselves smooth manifolds, allowing the construction of connections, etc. (whereas the set of smooth maps $\cal M \to V$ does not come with a manifold topology).
}
The gauge transformations of $\bm\psi|_p$ at a point $p$ form a group $G$ under composition---but this group does not describe local gauge transformations which act on the \emph{entire} field $\bm\psi$.
Rather, the total space of local gauge transformations is a principle $G$-bundle $\fibrebundle{G}{\cal G}{\cal M}$, consisting of smooth maps $\cal M \to G$.\footnote{
	We sometimes loosely refer to the fibre group $G$ as the gauge group, but strictly this means the fibre bundle $\cal G$ (following \cite{Tong_lecture_notes}).
}
The action of $\cal G$ on the space of matter fields $\cal E$ may be denoted
\begin{align}
	\bm\psi \overset{g}{\mapsto} \bm\psi' = g\cdot\bm\psi
\end{align}
for $\bm\psi \in \cal E$ and $g \in \cal G$ (i.e., $\bm\psi : \cal M \to V$ and $g : \cal M \to G$).
This specification of an action ``$\cdot$'' of $G$ on the vector space $V$ is equivalent to the prescription of a linear representation $\rho : G \to \mathrm{GL}(V)$.
Thus, we may represent the group action as matrix product,
\begin{math}
	% (g \cdot \bm\psi)|_p = \rho(g|_p)\bm\psi|_p
	g \cdot \bm\psi = \rho(g)\bm\psi
,\end{math}
remembering that both $\bm\psi$ and $g$ (but not $\rho$) vary across $\cal M$.

Here, it is worth noting that all gauge theories are equivalent to a gauge theory possessing only one matter field $\bm\psi_\text{total}$ with one gauge group $(G, \rho)$.
For instance, if a theory consists of different fundamental particles and forces, represented by fields $\bm\phi$ and $\bm\varphi$ with gauge groups $(G_\phi, \rho_\phi)$ and $(G_\varphi, \rho_\varphi)$,
% If a theory consists of qualitatively distinct fields $\bm\phi$ and $\bm\varphi$ with gauge groups $(G_\phi, \rho_\phi)$ and $(G_\varphi, \rho_\varphi)$, representing distinct fundamental particles and forces,
then we may take the total matter field $\bm\psi_\text{total} = \bm\phi \oplus \bm\varphi$ to be the direct sum of the fields in the theory, and similarly equip it with the gauge group $G_\text{total} = G_\phi \oplus G_\varphi$ with representation $\rho_\text{total} = \rho_\phi \oplus \rho_\varphi$.
This new theory is physically identical to the original.
In full generality, therefore, we treat our theory as possessing one matter and one gauge field, while freely speaking of its composite parts as separate fields where convenient.

A \emph{connection} on $\cal E$ is a derivation\footnotemark{} $\ts\nabla : \secs(\cal E) \to \forms^1(\cal M, \cal E)$ from vector fields to vector-valued 1-forms, designed so that $(\ts\nabla\bm\psi)(X)$---that is, the action of the 1-form $\ts\nabla\bm\psi$ on a vector $\ts X \in \T\cal M$---gives the directional derivative of $\bm\psi$ along $\ts X$ with respect to $\ts\nabla$.
\footnotetext{
	More precisely, $\ts\nabla$ is a $\cal C^\infty(\cal M)$-linear derivation, meaning
	\begin{math}
		\ts\nabla(f\bm u + g\bm v) = f\ts\nabla\bm u + g\ts\nabla\bm v
	\end{math}
	for scalar fields $f,g \in \cal C^\infty(\cal M)$ and
	\begin{math}
		\ts\nabla(\bm u \otimes \bm v) = \ts\nabla(\bm u) \otimes \bm v + \bm u \otimes \ts\nabla(\bm v)
	.\end{math}
}
Employing a basis $\qty{\bm e_a}$ of $\cal E$ and local coordinates $\qty{x^\mu}$ of $\cal M$, any connection is of the form
\begin{align}
	\ts\nabla \bm\psi &= \ts\dd\bm\psi + \tsbm A\bm\psi
\\	&= \qty(\partial_\mu\psi^a{}_b + A_\mu{}^a{}_b\psi^b) \, \ts\dd x^\mu \otimes \bm e_a
\end{align}
for some matrix-valued 1-form $\tsbm A$.
If the connection is required to transform like the matter field $\bm\psi$ under local gauge transformations, that is, as
\begin{align}
	\ts\nabla\bm\psi \overset{g}{\mapsto} g\cdot(\ts\nabla\bm\psi)
	= (g\cdot\ts\nabla)(g_\rho\bm\psi)
	= g_\rho\ts\nabla\bm\psi
,\end{align}
then $\ts\nabla$ is called a \emph{covariant derivative} and the connection 1-form $\tsbm A$ consequently obeys
\begin{align}
	\tsbm A \overset{g}{\mapsto} g\cdot\tsbm A = g_\rho\tsbm A g_\rho^{-1} - \qty(\ts\dd g_\rho)g_\rho^{-1}
.\end{align}

Such a connection $\ts\nabla_{\!A}$ is not unique; it depends on the choice of the 1-form field $\tsbm A$.
It is exactly this connection 1-form which is promoted to a dynamical object in the theory and given the name \emph{the gauge field}.
In order for the gauge field to be incorporated in the theory's equations of motion, we must have a notion of its derivative.
However, $\ts\nabla_{\!A} : \secs(\cal E) \cong \forms^0(\cal M, \cal E) \to \forms^1(\cal M, \cal E)$ is not defined on 1-forms such as $\tsbm A$ until we make the natural extension\footnotemark{} to a \emph{covariant exterior derivative}
% Finally, the connection $\ts\nabla : \secs(\cal E) \cong \forms^0(\cal M, \cal E) \to \forms^1(\cal M, \cal E)$ may be naturally extended\footnotemark{} to a \emph{covariant exterior derivative}
$\ts\dd_A : \forms^p(\cal M, \cal E) \to \forms^{p + 1}(\cal M, \cal E)$
\begin{align}
	% \ts\dd_A &: \forms^p(\cal M, \cal E) \to \forms^{p + 1}(\cal M, \cal E)
% ,&	\ts\dd_A &= \ts\dd + \tsbm A	
	\ts\dd_A &= \ts\dd + \tsbm A
% \\	\ts\dd_A\psi^a &= \ts\dd\psi^a + \ts A^a{}_b\psi^b
	% \ts\dd_A\tsbm\phi &= \ts\dd\tsbm\phi + \tsbm A\tsbm\phi
% \\	\qty(\partial_\mu\phi_\nu{}^a + A_\mu{}^a{}_b\phi_\nu{}^b) \, \ts\dd x^\mu \otimes \bm e_a
,\end{align}
\footnotetext{
	This is done by requiring the graded Leibniz property
	\begin{math}
		\ts\dd_A(\ts\phi\otimes\bm\psi) = \ts\dd\ts\phi \otimes\bm\psi + (-1)^p\ts\phi \wedge \ts\nabla\bm\psi
	\end{math}
	for $\ts\phi \in \forms^p(\cal M)$ and $\bm\psi \in \secs(\cal E)$, analogous to the usual exterior derivative.
}%
to allow the construction of, among other things, the curvature 2-form, or \emph{gauge field strength}
\begin{align}
	\tsbm F &\coloneq \ts\dd_A\tsbm A
\\	&= \ts\dd\tsbm A + \tsbm A \wedge \tsbm A
	= (\ts\dd\ts A^a{}_b + \ts A^a{}_c \wedge \ts A^c{}_b) \, \bm e_a \otimes \bm e^b
,\end{align}
where we have written the r.h.s.\ with respect to a basis $\qty{\bm e_a}$ and dual basis $\qty{\bm e^a}$ of $\cal E$.
The field strength $\tsbm F$ is useful because it is geometrical---that is, it transforms in the same matter as the matter fields $\tsbm F \mapsto g\cdot\tsbm F = g_\rho\tsbm F g_\rho^{-1}$, even though $\tsbm A$ does not.

\note{The gauge field strength also satisfies the Bianchi identity $\ts\dd_A\tsbm F = 0$.}

We have seen how the gauge field $\tsbm A$ and its strength $\tsbm F$ arise when we require the notion of a spacetime derivative for a matter field which possesses gauge freedom, and are now acquainted with the dynamical objects of a gauge theory.
The matter field $\bm\psi$, its derivative $\ts\nabla_{\!A}\bm\psi$ and the strength of the gauge field $\tsbm F$ are constructions which all transform regularly under gauge transformations.
All that remains to be specified in our theory are the equations of motion, which are to be expressed in terms of these three geometrical objects in a gauge invariant manner.


% Since $\ts\dd g : \cal M \to \frak g$ has values in the Lie algebra of $G$, the object $\ts\dd g_\rho = \ts\dd\rho(g)$ has values in $\dd\rho(\frak g)$
% From its transformation law, the gauge field $\tsbm A \in \forms^1(\cal E, \dd\rho(\frak g))$ may be seen to have values in the representation $\dd\rho$ of the Lie algebra $\frak g$ of $ G$, where the representation is that induced by the group representation $\rho$.


% There is still freedom left in the choice of connection, however, and it is exactly this object $\tsbm A$ which is promoted to a dynamical field and named the \emph{gauge field}.

% All that remains in our theory to specify are the equations of motion for $\bm\psi$ and $\tsbm A$.


% \section{The Yang--Mills Lagrangian}

% After making reasonable physical assumptions about a theory, there are only a few consistent possible terms constructed using only the gauge field strength $\tsbm F$ which may be included in the theory's Lagrangian.

\subsection{Lagrangians in QFT}
\label{sec:Lagrangians}

% We now have constructions for the matter field $\bm\psi$, its derivative $\ts\nabla_{\!A}\bm\psi$ and the strength of the gauge field $\tsbm F$ which all transform regularly under gauge transformations.
% All that remains to be specified in our theory are the equations of motion, which are to be expressed in terms of these three geometrical objects in a gauge invariant manner.

For classical gauge theories, the equations of motion may be specified as the Euler--Lagrange equations of a Lagrangian
\begin{align}
	L\qty[\bm\psi, \ts\nabla_{\!A}\bm\psi, \tsbm F] = \int_{\cal M}\! \LL\qty[\bm\psi, \ts\nabla_{\!A}\bm\psi, \tsbm F]
,\end{align}
where $\LL$ is a local Lagrangian density.
In order that the equations of motion are physically consistent, we require the Lagrangian possesses important symmetries: gauge symmetry, so that the equations of motion are gauge invariant; and Lorentz symmetry (which is automatic if $\LL = \mathscr{L}\,\vol$ is expressed as a volume form) \cite[§\,7.1]{Hamilton_2017}.
% We let the Lagrangian depend on derivatives only of order $\le 1$ in the fields for reasons of respecting causality and ensuring that energy has lower-bound.
The equations of motion are invariant under adjustments to the Lagrangian density by a total derivative $\ts\dd\alpha$, since by Stokes' theorem these contribute only to boundary terms.
Therefore, a Lagrangian which possesses gauge symmetry is still generally permitted to transform as
\begin{math}
	\LL \mapsto \LL + \ts\dd\alpha
\end{math}
under gauge transformations.

% The condition that Lagrangians of quantum field theories are , on the other hand, enjoy slightly enlarged symmetri, because of their origin in a path integral.
Furthermore, the Lagrangian of a quantum field theory enjoys an enlarged criterion of gauge symmetry: it is also permitted to transform as
\begin{align}
	\LL \mapsto \LL + n\cdot2\pi\hbar
	\label{eqn:discrete-QFT-Lagrangian-transformation}
,\end{align}
where $n \in \mathds Z$ may vary discretely under different gauges.
This is because of the origin of a QFT's equations of motion in the path integral.
For instance, the quantum mechanical amplitude that the fields $\bm\psi$ and $\tsbm F$ satisfy prescribed boundary conditions on $\partial\Omega$ surrounding some region of spacetime $\Omega \subseteq \cal M$ is given by the path integral
\begin{align}
	\mathscr A = \int_{\partial\Omega}\! \cal D[\bm\psi, \tsbm F] \exp\qty{\frac{i}{\hbar} \int_{\Omega}\! \LL\qty[\bm\psi, \ts\nabla_{\!A}\bm\psi, \tsbm F]}
	\label{eqn:path-integral}
,\end{align}
where the intended meaning of $\cal D[\cdots]$ is an integration over all field configurations on $\Omega$.
If the Lagrangian were to undergo a gauge transformation \eqref{eqn:discrete-QFT-Lagrangian-transformation}, the amplitude $\mathscr A \mapsto \mathscr A\exp(n\cdot2\pi i)$ would be left invariant.
In other words, physical consistency of a QFT does not require the single-valuedness of $\LL$, but only of $\exp(i\LL/\hbar)$.


% Another condition on the Lagrangian in QFT comes from the expectation that the quantised theory be \emph{renormalisable}
We also require that the quantum field theory associated to a Lagrangian be \emph{renormaliseble}.
This restricts the form of the Lagrangian considerably.
% For instance, in 4-dimensional spacetime, can be shown that

\note{transition}

\subsection{The Dirac Lagrangian}

\subsection{The Yang--Mills Lagrangian and the Topological \texorpdfstring{$\theta$-Term}{θ-Term}}
\label{sec:YM-and-topological}

Taking Lorentz and gauge invariance into account, the only admissible field Lagrangians which may be constructed from the gauge field $\tsbm A$ alone are linear combinations of
\begin{align}
	\ang{\tsbm F \wedge \star\tsbm F} &\equiv \frac12\angg{\bm F_{\mu\nu}, \bm F^{\mu\nu}}\vol
,&	&\text{and}
&	\ang{\tsbm F \wedge \tsbm F} &\equiv \frac14\angg{\bm F_{\mu\nu}, \bm F_{\rho\sigma}}\epsilon^{\mu\nu\rho\sigma}\vol
,\end{align}
where $\star$ is the Hodge dual and $\angg{\ ,\ }$ is an inner product on the Lie algebra $\frak g$ of the gauge group $G$ (recall that $\tsbm F$ is a $\frak g$-valued form) \cite[§\,7.1.2]{Hamilton_2017}.
The inner product $\angg{\ ,\ }$ on $\frak g$ is not unique; it depends on a choice of \emph{coupling constants}.
In particular, if the gauge group $G$ is the direct sum of $n$ simple Lie groups,%
\footnote{
	Any compact connected Lie group is either of this form, or is a finite quotient of such a group, if $\mathrm U(1)$ is counted as `simple'. \cite[§\,2.4.3]{Hamilton_2017}.
}
then $\angg{\ ,\ }$ is specified by the choice of exactly $n$ coupling constants, one corresponding to each factor of $G$.
Physically, the coupling constants determine the relative interaction strengths of the forces associated to each factor of $G$ \cite[§\,2.5]{Hamilton_2017}
(For instance, the gauge group of the standard model \eqref{eqn:SM-gauge-group} has three such coupling constants for the strong, weak, and electromagnetic forces.)

The term $\ang{\tsbm F \wedge\star \tsbm F}$ is known as the \emph{Yang--Mills Lagrangian}, and is a major component in the standard model, describing boson force carrier self-interactions (such as gluon self-interactions).
The Yang--Mills Lagrangian yields the equations of motion $\ts\dd_A\star\tsbm F = 0$.
For the Abelian gauge group of electromagnetism $G = \mathrm{U}(1)$, these, together with the Bianchi identity $\ts\dd_A\tsbm F = 0$, are the source-free Maxwell's equations.


The other term $\ang{\tsbm F \wedge \tsbm F}$ is known as the \emph{topological $\theta$-term} for reasons that will shortly become apparent.
The $\theta$-term breaks both time-reversal symmetry $T$ and parity $P$ (since $\epsilon^{\mu\nu\rho\sigma} \mapsto -\epsilon^{\mu\nu\rho\sigma}$ under $T$ or $P$).
It is \emph{topological} because its integral over spacetime is discrete;
\begin{align}
	\frac{1}{8\pi^2}\int_{\cal M} \ang{\tsbm F \wedge \tsbm F} = n_A \in \mathds{Z}
,\end{align}
where $n_A$ is a topological invariant of the configuration of the gauge field $\tsbm A$, known as the Pontryagin number (or informally, the `winding number') \cite[§\,2.2]{Tong_lecture_notes}.
\note{Needs more explanation. Relates to instanton configurations.}
% This is because it is in fact the derivative of the \emph{Chern--Simons 3-form} $\CS$, ($\ang{\tsbm F\wedge\tsbm F} = \ts\dd\CS$, incidentally meaning t),


% For Abelian gauge groups such as $\mathrm{U}(1)$, the topology of the space of gauge fields, $\forms^1(\cal M, \mathfrak g)$, is trivial (in the sense that it is path-connected to the identity) and the Pontryagin number $n_A$ is zero for all $\tsbm A$.
% However, for general gauge groups, $\forms^1(\cal M, \mathfrak g)$ may have non-trivial topology.

The $\theta$ in the name refers to the cyclic property of a coefficient $\theta$ given to the topological $\theta$-term.
If the Lagrangian density includes the term
\begin{align}
	\frac{\theta}{8\pi^2}\ang{\tsbm F\wedge\tsbm F}
,\end{align}
then its contribution to the action in the path-integral \eqref{eqn:path-integral} is $\theta n_A$, and since
\begin{math}
	\exp(i\theta n_A) = \exp(i(\theta + 2\pi)n_A)
,\end{math}
the coefficient $\theta \in [0, 2\pi)$ is naturally periodic.




\section{QCD and the Strong \CP\ Problem}

Quantum chromodynamics is a quantum gauge theory describing the strong interactions between hadronic matter, whose matter field describes quarks, and the gauge field describes gluons.
Quarks are fermions described by Dirac fields which come in pairs of \emph{up type} and \emph{down type} quarks, each transforming under different representations of weak hypercharge $\mathrm{U}(1)$ (but which are identical in pure QCD).
QCD herself is agnostic to the number of quark pairs, but when included in the standard model, three copies (each called a \emph{generation}) of quark pairs are used, meaning the standard model contains six quark fields in total .

One of the attractive qualities of pure QCD is that, given the gauge group $G = \mathrm{SU}(3)$ and its action on a matter field $\bm\psi$, the theory is almost completely specified given reasonable physical assumptions.
That is, there is not much freedom for QCD to be slightly different without being an essentially different theory.

The Lagrangian of QCD is a sum of the Dirac Lagrangian and the Yang--Mills Lagrangian:
\begin{align}
	% \cal L_\text{QCD} = \overbar{\bm\psi}\qty(? - m)\bm\psi - \ang{\tsbm F \wedge\star \tsbm F}
	\cal L_\text{QCD} &= \overbar{\bm\psi}\qty(i\bm\gamma^\mu\nabla_\mu - m)\bm\psi - \ang{\tsbm F \wedge\star \tsbm F}
\\	&\equiv \overbar{\psi}_a\qty(i\gamma^{\mu a}{}_b\nabla_\mu - m\delta^a{}_b)\psi^b - \frac14F^a{}_{b\mu\nu}F_a{}^{b\mu\nu}
\end{align}


At this point, we see that the issue which is given the name \emph{the strong \CP\ problem} offers itself as a question about the naturalness the QCD Lagrangian.
As encountered in section~\ref{sec:YM-and-topological}, there is another `natural' term that may be included; the topological $\theta$-term.
In this sense, the most general QCD Lagrangian is
\begin{align}
	\cal L_\text{QCD} = \overbar{\bm\psi}\qty(i\bm\gamma^\mu\nabla_\mu - m)\bm\psi
	- \ang{\tsbm F \wedge\star \tsbm F}
	+ \frac{\theta}{8\pi^2}\ang{\tsbm F \wedge \tsbm F}
,\end{align}
which includes the \CP-violating $\theta$-term $\ang{\tsbm F\wedge\tsbm F}$ with a coefficient named the $\theta$-parameter.
\note{`Natural' here means that we lack a reason that the $\theta$-term should not contribute, unlike the case of QED, where ``there are no instantons''.}
Especially given the fact that \CP\ symmetry is not a feature of the standard model, there is no reason to exclude the \CP-violating term in the QCD Lagrangian.


If the $\theta$-parameter is nonzero, then the strong force is predicted to violate \CP\ in general.
A significant implication of \CP\ violation is that the neutron is expected to possess an electric dipole moment of approximate magnitude $|d_n| \approx 10^{-18} \,e\,\mathrm{cm}$.
In reality, current measurements \cite{electric_dipole_neutron_2020} of the neutron's electric dipole moment place an upper bound of $|d_n| \lesssim 10^{-26} \,e\,\mathrm{cm}$, and this implies a stringent constraint on the responsible $\theta$-term, $|\theta| \lesssim 10^{-10}$ \cite{Review_2018}.

