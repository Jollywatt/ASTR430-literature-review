\chapter{QCD and the Strong \CP\ Problem}

\textsc{The Standard Model} of particle physics is a quantum field theory which describes all known fundamental particles and interactions to very high precision---with notable exceptions including:
the absence of gravity;
dark matter;
and the experimentally incorrect prediction that neutrinos are massless.
Alongside these shortcomings, the standard model also exhibits issues of a more philosophical nature, such as the reasons for the values of the theory's many free parameters. (For example; why are there three generations of particles? Why do the particle masses and coupling constants have the values they do?)
Among these mysteries is the \emph{strong \CP\ problem}, a fine-tuning problem regarding the apparent symmetry of the strong force under time reversal.\footnote{Time reversal $T$ and charge--parity $CP$ symmetry are equivalent assuming the combined charge--parity--time symmetry; i.e., $T = CP \mod CPT$.}
This chapter summarises the background theory necessary to understanding the statement and origin of the strong \CP\ problem.

The standard model was born amid the explosion of phenomenological particle physics in the early 1960s, in an effort to explain the patterns in the rapidly increasing catalogue of known particles.
The introduction of the quark model explained the properties of numerous hadrons in terms of six kinds of quark, and was given the name \emph{quantum chromodynamics} (QCD). This, together with the well-established quantum theory of electromagnetism, \emph{quantum electrodynamics} (QED), became the first approximation to the standard model \cite{Weinberg_2004-history}.

The modern statement of the standard model is that it is a Lorentz-invariant \emph{quantum field theory} whose \emph{gauge group} is the compact connected Lie group
\begin{align}
	\qty\big(\SU(3)\oplus\SU(2)\oplus\mathrm{U}(1))\big/\mathds{Z}_6
	\label{eqn:SM-gauge-group}
,\end{align}
which is equipped with a particular action on matter fields.
A quantum field theory is a quantised \emph{gauge field theory}, which is a field theory whose Lagrangian is symmetric under the action of the gauge group (elaborated upon in §\,\ref{sec:overview-of-gauge-theory}).
Each term in the gauge group of the standard model \eqref{eqn:SM-gauge-group} corresponds to a fundamental force of nature:
The factor $\SU(3)$ is the gauge group of quantum chromodynamics, the sector of the standard model describing the interactions of colour-charged quarks and gluons via the strong force.
The factor $\SU(2)\oplus\mathrm{U}(1)$ corresponds to the unified electroweak force, and contains another $\mathrm{U}(1)$ sub-factor corresponding to the electromagnetic force of QED.

Quantum chromodynamics was formulated quite some time after quantum electrodynamics was understood, which is a reflection that QCD is more intricate than QED.
Unlike QED, quantum chromodynamics exhibits \emph{asymptotic freedom}, meaning that the effective strength of the strong interaction between colour-changed particles \emph{increases} with increasing separation.
Thus, QCD is severely non-perturbative and evades even modern analytical treatment, except at low energies.
Asymptotic freedom also results in quark confinement, which contributed to the confusion of particle physicists in the 1960s---quarks could never be observed in isolation.
The dissimilarity of QCD to QED can be credited largely to the fact that the gauge group $\SU(3)$ is non-Abelian, corresponding to the fact that the gluon force carriers of QCD are themselves colour-charged, which results in asymptotic freedom.

However, quantum chromodynamics remains a highly successful and beautiful theory.
% It is successful in the sense that offers precise predictions
Aside from the technical problems of its non-perturbative nature, QCD also exhibits a mystery of purely theoretical concern: the \emph{strong \CP\ problem}.
% This is a problem of fine-tuning and is a smoking-gun for physics beyond the standard model.
This problem is of the fine-tuning variety, but it lacks even an anthropic solution and is good reason to pursue physics beyond the standard model.


The significance and role of the gauge group is outlined in the next section where a geometrical overview of gauge field theory is presented.
This overview is intended to equip the reader with a geometrical idea of the kinds of objects which comprise classical gauge theories and which underpin the standard model.
Then, QCD is defined and the origin of the so-called \emph{strong \CP\ problem} is highlighted.


\note{Justify focus on classical gauge theory}

\section{Overview of Classical Gauge Theory}
\label{sec:overview-of-gauge-theory}

Gauge theories take place on a spacetime manifold and consist of two mathematically distinct dynamical entities: a \emph{matter field} and a \emph{gauge field}.
A classical gauge theory is completely specified by the prescription of four ingredients: the base manifold; the vector space in which the matter field takes its values; a \emph{gauge group} equipped with an action on the matter field; and equations of motion, usually provided by a Lagrangian density.
The dynamical gauge field is not prescribed at the outset---it arises naturally as a consequence of the matter field's symmetry under the action of the gauge group.

The underlying premise of gauge theory is that there may be physical redundancy in the mathematical description of a matter field at each point in spacetime.
(For example, the complex phase of a wavefunction is physically irrelevant.)
This redundancy results in mathematical degrees of freedoms of the matter fields which are non-physical, called \emph{local gauge freedoms}.
(Keeping with our example: local rotations of phase are a gauge freedom.)
Crucially, local gauge freedoms introduce ambiguity in the notion of the physical rate of change of matter fields about a point, because the point-wise independence of its gauge freedom means that the difference in the matter field between nearby points is gauge-dependent.
In other words, there is no preferred directional derivative of a matter field with local gauge freedom, until the choice of a \emph{connection} is made.
The triumph of gauge theory is that, by introducing the gauge field to act as a connection, this ambiguity is promoted to a separate set of physical degrees of freedom.
The implications of this are twofold:
Firstly, the gauge field yields a choice of derivative (namely, the covariant derivative with respect to the gauge field), allowing the inclusion of well-defined derivatives of matter fields in the theory's equations of motion.
Secondly, the gauge field has a dynamical role in the theory, describing a new kind of field: force fields (and, after quantisation, the force carrier bosons).



\begin{figure}
\begin{aside}[Notations used in this chapter]
	\begin{tabular}{cl}
		% $\hookrightarrow$; $\twoheadrightarrow$; $\leftrightarrow$
		$\cal{A, B, ...}$
	&	objects with manifold structure
	\\	$\fibrebundle{}{;\,}{};\,\leftrightarrow$
	&	injection; surjection; bijection
	\\	$\fibrebundle[\pi]{F}{\cal E}{\cal M}$
	&	fibre bundle $\cal E$ over base space $\cal M$ with fibre $F$ and projection $\pi : \cal E \twoheadrightarrow \cal M$
	\\	$\T_p\cal M$; $\T\cal M$
	&	tangent space at $p \in \cal M$; tangent bundle of $\cal M$
	\\	$\T^r_s\cal M$
	&	type $\smqty(r\\s)$ tensors over $\cal M$, equal to
		\begin{math}
			\qty(\bigotimes_{i=1}^r\T\cal M) \otimes \qty(\bigotimes_{i=1}^s\T^*\cal M)
		\end{math}
	\\[.7ex]	$\TT\cal M$
	&	tensor bundle of $\cal M$, equal to
		\begin{math}
			% \bigoplus_{r,s\in\mathds N}\T^r_s\cal M
			\bigoplus_{r,s=0}^\infty\T^r_s\cal M
		\end{math}
	\\	$\ts A$
		% & Spacetime tensor; an element of $\T\cal M\otimes\cdots\otimes\T\cal M\otimes\T^*\cal M\otimes\cdots\otimes\T^*\cal M$
	&	spacetime tensor field; i.e., a section of $\TT\cal M$
	\\	$\bm A$
	&	vector field of some abstract vector space not contained in $\TT\cal M$
	\\	$\tsbm A$
	&	spacetime tensor field with values in some other abstract vector space
	\\	$\wedge^pV$
	&	$p$th exterior power of the vector space $V$; i.e., the space of $V$-valued $p$-forms
	\\	$\secs(\cal E)$
	&	smooth sections $\cal M \to F$ of fibre bundle $\fibrebundle{F}{\cal E}{\cal M}$
	\\	$\forms^p(\cal M)$
	&	space of $p$-forms on $\cal M$, equal to $\secs(\wedge^p \T^*\cal M)$
	\\	$\forms^p(\cal M, V)$
	&	space of $V$-valued $p$-forms, equal to $\secs(\cal E\otimes\wedge^p \T^*\cal M)$ where $\fibrebundle{V}{\cal E}{\cal M}$
	% \\	$\frak g$
	% &	Lie algebra of group $G$
	% \\	$\dd$
	% &	exterior derivative
	% \\	$\nabla_{\tsbm A}$
	% &	covariant derivative with respect to $\tsbm A$
	% \\	$\dd_A$
	% &	covariant exterior derivative with respect to $\tsbm A$
	\end{tabular}
\end{aside}
\end{figure}

In geometrical language, the matter field $\bm ψ \in \secs(\cal E)$ is a section of a vector bundle $\fibrebundle{V}{\cal E}{\cal M}$.
In other words, at any point $p \in \cal M$ in spacetime the field assigns a vector, $\bm ψ|_p \cong V$.%
\footnote{
	The reason many objects in gauge theory (such as fields $\bm ψ$) are defined as sections of fibre bundles (and not simply as smooth maps $\bm ψ : \cal M \to V$) is because fibre bundles are themselves smooth manifolds which admit the construction of connections.
}
The gauge transformations of $\bm ψ|_p$ at a point $p$ form a group $G$ under composition---but this group does not describe local gauge transformations which act on the \emph{entire} field $\bm ψ$.
Rather, the total space of local gauge transformations is a principle $G$-bundle $\fibrebundle{G}{\cal G}{\cal M}$, consisting of smooth maps $\cal M \to G$.\footnote{
	We sometimes loosely refer to the fibre group $G$ as the gauge group, but strictly this means the fibre bundle $\cal G$ (following \cite{Tong_lecture_notes}).
}
The action of $\cal G$ on the space of matter fields $\cal E$ may be denoted
\begin{align}
	\bm ψ \overset{g}{\mapsto} \bm ψ' = g\cdot\bm ψ
\end{align}
for $\bm ψ \in \secs(\cal E)$ and $g \in \secs(\cal G)$ (i.e., $\bm ψ : \cal M \to V$ and $g : \cal M \to G$).
This specification of an action ``$\cdot$'' of $\cal G$ on the vector space $\cal E$ is equivalent to the choice of a linear representation $\rho : G \to \mathrm{End}(V)$ of $G$ on $V$.
Thus, we may represent the group action as matrix product,
\begin{math}
	% (g \cdot \bm ψ)|_p = \rho(g|_p)\bm ψ|_p
	g \cdot \bm ψ = \rho(g)\bm ψ
,\end{math}
remembering that both $\bm ψ$ and $g$ (but not $\rho$) vary across $\cal M$.

Here, it is worth noting that all gauge theories are isomorphic to a gauge theory possessing only one matter field $\bm ψ_\text{total}$ with one gauge group $(G, \rho)$.
For instance, if a theory consists of different fundamental particles and forces, represented by fields $\bm\phi$ and $\bm\varphi$ with gauge groups $(G_\phi, \rho_\phi)$ and $(G_\varphi, \rho_\varphi)$,
% If a theory consists of qualitatively distinct fields $\bm\phi$ and $\bm\varphi$ with gauge groups $(G_\phi, \rho_\phi)$ and $(G_\varphi, \rho_\varphi)$, representing distinct fundamental particles and forces,
then we may take the total matter field $\bm ψ_\text{total} = \bm\phi \oplus \bm\varphi$ to be the direct sum of the fields in the theory, and similarly equip it with the gauge group $G_\text{total} = G_\phi \oplus G_\varphi$ with representation $\rho_\text{total} = \rho_\phi \oplus \rho_\varphi$.
This new theory is physically identical to the original.
In full generality, therefore, we treat our theory as possessing one matter and one gauge field, while freely speaking of its composite parts as separate fields where convenient.

A \emph{connection} on $\cal E$ is a derivation\footnotemark{} $\ts\nabla : \secs(\cal E) \to \forms^1(\cal M, \cal E)$ from vector fields to vector-valued 1-forms, designed so that $(\ts\nabla\bm ψ)(\ts X)$---that is, the action of the $V$-valued 1-form $\ts\nabla\bm ψ$ on a vector $\ts X \in \T\cal M$---gives the directional derivative of $\bm ψ$ along $\ts X$ with respect to $\ts\nabla$.
\footnotetext{
	More precisely, $\ts\nabla$ is a $\cal C^\infty(\cal M)$-linear derivation, meaning
	\begin{math}
		\ts\nabla(f\bm u + g\bm v) = f\ts\nabla\bm u + g\ts\nabla\bm v
	\end{math}
	for scalar fields $f,g \in \cal C^\infty(\cal M)$ and
	\begin{math}
		\ts\nabla(\bm u \otimes \bm v) = \ts\nabla(\bm u) \otimes \bm v + \bm u \otimes \ts\nabla(\bm v)
	.\end{math}
}
Employing a basis $\qty{\bm e_a}$ of $\cal E$ and local coordinates $\qty{x^μ}$ of $\cal M$, \emph{any} connection is of the form
\begin{align}
	\ts\nabla \bm ψ &= \ts\dd\bm ψ + \tsbm A\bm ψ
\\	&= \qty(\partial_\muψ^a{}_b + A_μ{}^a{}_bψ^b) \, \ts\dd x^μ \otimes \bm e_a
\end{align}
for some matrix-valued 1-form $\tsbm A$, where $\ts\dd$ is the exterior derivative.
If $\ts\nabla\bm ψ$ is required to transform like the matter field $\bm ψ$ under local gauge transformations, that is, as
\begin{align}
	\ts\nabla\bm ψ \overset{g}{\mapsto} g\cdot(\ts\nabla\bm ψ)
	= (g\cdot\ts\nabla)(g_\rho\bm ψ)
	= g_\rho\ts\nabla\bm ψ
,\end{align}
then $\ts\nabla$ is called a \emph{covariant derivative} and the connection 1-form $\tsbm A$ consequently obeys
\begin{align}
	\tsbm A \overset{g}{\mapsto} g\cdot\tsbm A = g_\rho\tsbm A g_\rho^{-1} - \qty(\ts\dd g_\rho)g_\rho^{-1}
.\end{align}

Such a connection $\ts\nabla_{\!A}$ is not unique; it depends on the choice of the 1-form field $\tsbm A$.
It is exactly this connection 1-form which is promoted to a dynamical object in a gauge theory and given the name \emph{the gauge field}.
In order for the gauge field to be incorporated in the theory's equations of motion, we would like to have a notion of its derivative.
However, $\ts\nabla_{\!A} : \secs(\cal E) \cong \forms^0(\cal M, \cal E) \to \forms^1(\cal M, \cal E)$ is not defined on 1-forms such as $\tsbm A$ until we make the natural extension\footnotemark{} to a \emph{covariant exterior derivative}
% Finally, the connection $\ts\nabla : \secs(\cal E) \cong \forms^0(\cal M, \cal E) \to \forms^1(\cal M, \cal E)$ may be naturally extended\footnotemark{} to a \emph{covariant exterior derivative}
$\covdd : \forms^p(\cal M, \cal E) \to \forms^{p + 1}(\cal M, \cal E)$
\begin{align}
	% \covdd &: \forms^p(\cal M, \cal E) \to \forms^{p + 1}(\cal M, \cal E)
% ,&	\covdd &= \ts\dd + \tsbm A	
	\covdd &= \ts\dd + \tsbm A
% \\	\covddψ^a &= \ts\ddψ^a + \ts A^a{}_bψ^b
	% \covdd\tsbm\phi &= \ts\dd\tsbm\phi + \tsbm A\tsbm\phi
% \\	\qty(\partial_μ\phi_ν{}^a + A_μ{}^a{}_b\phi_ν{}^b) \, \ts\dd x^μ \otimes \bm e_a
,\end{align}
\footnotetext{
	% Note that $\secs(\cal E) = \forms^0(\cal M, \cal E)$.
	The extension is obtained by requiring the graded Leibniz property
	\begin{math}
		\covdd(\ts\phi\otimes\bm ψ) = \ts\dd\ts\phi \otimes\bm ψ + (-1)^p\ts\phi \wedge \ts\nabla\bm ψ
	\end{math}
	for $\ts\phi \in \forms^p(\cal M)$ and $\bm ψ \in \secs(\cal E)$, analogous to the usual exterior derivative.
}%
to allow the construction of, among other things, the curvature 2-form, or \emph{gauge field strength}
\begin{align}
	\tsbm F &\coloneq \covdd\tsbm A
\\	&= \ts\dd\tsbm A + \tsbm A \wedge \tsbm A
	= (\ts\dd\ts A^a{}_b + \ts A^a{}_c \wedge \ts A^c{}_b) \, \bm e_a \otimes \bm e^b
,\end{align}
where we have written the r.h.s.\ with respect to a basis $\qty{\bm e_a}$ and dual basis $\qty{\bm e^a}$ of $\cal E$ (note that $V$ is equipped with the Euclidean inner product; $\bm e^a (\bm e_b) = \delta^a_b$).
The field strength $\tsbm F$ is useful because it is geometrical, in the sense that it transforms like the matter field; $\tsbm F \mapsto g\cdot\tsbm F = g_\rho\tsbm F g_\rho^{-1}$ under gauge transformations, even though $\tsbm A$ does not.

\note{The gauge field strength also satisfies the Bianchi identity $\covdd\tsbm F = 0$.}

We have seen how the gauge field $\tsbm A$ and its strength $\tsbm F$ arise when we require the notion of a spacetime derivative for a matter field which possesses gauge freedom, and are now acquainted with the dynamical objects of a gauge theory.
The matter field $\bm ψ$, its derivative $\ts\nabla_{\!A}\bm ψ$ and the strength of the gauge field $\tsbm F$ are constructions which all transform regularly under gauge transformations.
All that remains to be specified in our theory are the equations of motion, which are to be expressed in terms of these three geometrical objects in a gauge invariant manner.


\subsection{Lagrangians in Field Theories}
\label{sec:Lagrangians}

For classical gauge theories, the equations of motion may be specified as the extremisation of an action
\begin{align}
	S\qty[\bm ψ, \ts\nabla_{\!A}\bm ψ, \tsbm F] = \int_{\cal M}\! \LL\qty[\bm ψ, \ts\nabla_{\!A}\bm ψ, \tsbm F]
,\end{align}
where $\LL$ is a local Lagrangian density.
In order that the equations of motion are physically well-defined, we require the Lagrangian possesses the relevant symmetries: gauge symmetry, so that the equations of motion are gauge invariant; and Lorentz symmetry (which is automatic if $\LL = \mathscr{L}\,\vol$ is expressed as a volume form) \cite[§\,7.1]{Hamilton_2017}.
% We let the Lagrangian depend on derivatives only of order $\le 1$ in the fields for reasons of respecting causality and ensuring that energy has lower-bound.
The equations of motion are invariant under adjustments to the Lagrangian density by a total derivative $\ts\dd\ts K$, since by Stokes' theorem these contribute only to boundary terms.
Therefore, a Lagrangian which is said to possess gauge symmetry is still generally permitted to transform as
\begin{align}
	\LL \mapsto \LL + \ts\dd\ts K
	\label{eqn:Lagrangian-transforming-by-exact-form}
\end{align}
under gauge transformations, if the fields vanish at the boundary.
If a Lagrangian transforms as \eqref{eqn:Lagrangian-transforming-by-exact-form} under a gauge symmetry parametrised by $n$ parameters $α_i$, then Noether's theorem implies the existence of $n$ conserved current densities (viz. 3-forms), one for each $α_i$,
\begin{align}
	\ts J_{(i)} = \pdv{\LL}{\ts\nabla\bm ψ} \eval{\fdv{\bm ψ}{α^i}}_{\mathrm{id}} - \ts K
,\end{align}
whose continuity equations read $\ts\dd\ts J_{(i)} = 0$.

% The condition that Lagrangians of quantum field theories are , on the other hand, enjoy slightly enlarged symmetri, because of their origin in a path integral.
Furthermore, the Lagrangian of a quantum field theory enjoys an enlarged criterion of gauge symmetry: it is also permitted to transform as
\begin{align}
	% \int\LL \mapsto \int\LL + n\cdot2\pi\hbar
	S \mapsto S + n\cdot2\pi\hbar
	\label{eqn:discrete-QFT-Lagrangian-transformation}
,\end{align}
where $n \in \mathds Z$ may vary discretely under different gauges.
This is because of the origin of a QFT's equations of motion in the Feynman path integral.
For instance, the quantum mechanical amplitude that the fields $\bm ψ$ and $\tsbm F$ satisfy prescribed boundary conditions on $\partial\Omega$ surrounding some region of spacetime $\Omega \subseteq \cal M$ is given by the path integral
\begin{align}
	\mathscr A = \int_{\partial\Omega}\! \cal D[\bm ψ, \tsbm F] \exp\qty{\frac{i}{\hbar} \int_{\Omega}\! \LL\qty[\bm ψ, \ts\nabla_{\!A}\bm ψ, \tsbm F]}
	\label{eqn:path-integral}
,\end{align}
where the intended meaning of $\cal D[\cdots]$ is an integration over all field configurations on $\Omega$.
If the Lagrangian were to undergo a discrete gauge transformation \eqref{eqn:discrete-QFT-Lagrangian-transformation}, the amplitude $\mathscr A \mapsto \mathscr A\exp(n\cdot2\pi i)$ would be left invariant.
In other words, physical consistency of a QFT does not require the single-valuedness of $S$, but only of $\exp(iS/\hbar)$.
This leads to an enlargement of the space of possible Lagrangian densities to include, in particular, topological terms.
These prove to be especially relevant to QFTs and to the strong \CP\ problem itself.


% Another condition on the Lagrangian in QFT comes from the expectation that the quantised theory be \emph{renormalisable}
% For instance, in 4-dimensional spacetime, can be shown that



\subsection{The Yang--Mills Lagrangian and the Topological \texorpdfstring{$θ$-Term}{θ-Term}}
\label{sec:YM-and-topological}

\note{We seek to describe the dynamics of the gauge field.}
Along with requiring Lorentz and gauge invariance, we also require that the quantum field theory associated to a Lagrangian be \emph{renormaliseble}.
Loosely speaking, a classical field theory is renormalisable if it can be ``quantised consistently''.
(This restricts the form of the Lagrangian considerably, but how this happens is beyond this review's scope.)
Under these constraints, the only admissible QFT Lagrangians which may be constructed from the gauge field $\tsbm A$ alone are linear combinations of
\begin{align}
	% \angbr{\tsbm F \wedge \star\tsbm F} &\equiv \frac12\angbrAd{\bm F_{μν}, \bm F^{μν}}\vol
	\angbr{\tsbm F \wedge \star\tsbm F} &\equiv \frac12\angbrAd{\bm F_{μν}, \bm F_{\rho\sigma}}g^{μ\rho}g^{ν\sigma}\vol
,&	&\text{and}
&	\angbr{\tsbm F \wedge \tsbm F} &\equiv \frac14\angbrAd{\bm F_{μν}, \bm F_{\rho\sigma}}\epsilon^{μν\rho\sigma}\vol
,\end{align}
where $\star$ is the Hodge dual \cite[§\,7.1.2]{Hamilton_2017}.
The inner product $\angbrAd{\ ,\ }$ on the Lie algebra $\frak g$ of the gauge group $G$ is chosen such that it is gauge-invariant.
(Recall that $\tsbm F$ is a $\frak g$-valued form, so an inner product on $\frak g$ is needed. \cite[§\,7.3]{Hamilton_2017}.) % also [Folland ch. 9]
The inner product $\angbrAd{\ ,\ }$ on $\frak g$ is not unique; it depends on a choice of \emph{coupling constants}.
In particular, if the gauge group $G$ is the direct sum of $n$ simple Lie groups,%
\footnote{
	Any compact connected Lie group is either of this form, or is a finite quotient of such a group, if $\mathrm U(1)$ is counted as `simple'. \cite[§\,2.4.3]{Hamilton_2017}.
}
then $\angbrAd{\ ,\ }$ is specified by the choice of exactly $n$ coupling constants, one corresponding to each factor of $G$ \cite[§\,2.5]{Hamilton_2017}.
Physically, the coupling constants determine the relative interaction strengths of the forces associated to each factor of $G$, and must enter the theory as free parameters determined experimentally.
(For instance, the gauge group of the standard model \eqref{eqn:SM-gauge-group} has three such coupling constants for the strong $\SU(3)$, weak $\SU(2)$, and electromagnetic $\mathrm{U}(1)$ interactions.)

% This construction was first considered by Yang and Mills in 1954 [135] in the special case where G = SU(2) and Φ is a pair of Dirac spinor fields. [Folland pg 295]
The term $\angbr{\tsbm F \wedge\star \tsbm F}$ is known as the \emph{Yang--Mills Lagrangian}, and is a major component in the standard model, describing boson force carrier propagation and self-interaction (such as gluon self-interactions).
The Yang--Mills Lagrangian yields the equation of motion $\covdd\star\tsbm F = 0$.
For the Abelian gauge group of electromagnetism $G = \mathrm{U}(1)$, this equation, together with the Bianchi identity $\covdd\tsbm F = 0$, are the source-free Maxwell's equations.


The other term $\angbr{\tsbm F \wedge \tsbm F}$ is known as the \emph{Chern--Simons term} or the \emph{topological $θ$-term}, for reasons which will become apparent.
The Chern--Simons term is odd under both time-reversal symmetry $T$ and parity $P$ (notice $\epsilon^{μν\rho\sigma} \mapsto -\epsilon^{μν\rho\sigma}$ under $T$ or $P$).
It is \emph{topological} because it does not depend on the geometry of spacetime via the metric $g^{μν}$ (instead, all spacetime indices are contracted with $\epsilon^{μν\rho\sigma}$).
Hence, an action $\int_{\cal M} \angbr{\tsbm F \wedge \tsbm F}$ depends only on the integrand's topology over $\cal M$.
In fact, the integral is a discrete topological invariant
\begin{align}
	n = \frac{1}{8\pi^2}\int_{\cal M} \angbr{\tsbm F \wedge \tsbm F} \in \mathds{Z}
,\end{align}
known as the \emph{Pontryagin number}, the \emph{second Chern class} \cite[§\,1]{Witten_1989} or simply the `winding number' \cite[§\,2.2]{Tong_lecture_notes} of the gauge field configuration $\tsbm A$.
Furthermore, $\angbr{\tsbm F \wedge \tsbm F}$ is a total derivative of the \emph{Chern--Simons 3-form} $\csform$,
\begin{align}
	\angbr{\tsbm F \wedge \tsbm F}
	= \ts\dd\csform
	= \ts\dd\tr\qty(\tsbm A\wedge\ts\dd\tsbm A + \frac{2}{3}\tsbm A\wedge\tsbm A\wedge\tsbm A)
,\end{align}
meaning that the action depends only on the topology of $\tsbm A$ on the spacetime boundary $\partial\cal M$.\footnote{
	As such, the $θ$-term does not affect classical equations of motion.
	However, in the quantum theory of fermions, it has important non-perturbative effects (an implication of the \emph{chiral anomaly}) \cite[§\,3]{Tong_lecture_notes}.
}


For an Abelian gauge group $\mathrm{U}(1)$, the space of gauge fields $\forms^1(\cal M, \mathfrak g)$ is path-connected to the identity, and the winding number $n$ is zero for all $\tsbm A \in \forms^1(\cal M, \mathfrak g)$.
Consequently, the Chern--Simons term is neither relevant in classical electromagnetism nor in QED.
However, for non-Abelian gauge groups such as the $\SU(3)$ of QCD, the space of gauge fields $\forms^1(\cal M, \mathfrak g)$ has a non-trivial topology,\footnotemark\ containing disjointed regions labelled by $n \in \mathds{Z}$.
\footnotetext{
	Technically, we consider the topology of the subset of $\forms^1(\cal M, \mathfrak g)$ consisting only of gauge fields $\tsbm A$ which approach zero at spatial infinity.
}
In other words, gauge configurations in QCD can only be continuously transformed into each other if they share the same winding number.
Gauge field configurations with $n \ne 0$, which live outside of the identity-connected component of $\forms^1(\cal M, \mathfrak g)$, are known as \emph{instanton configurations}.
\note{Actually, instantons must satisfy $\star\tsbm F = \pm\tsbm F$, and enable quantum tunnelling between homotopy classes.}
Instantons are an important aspect of QCD and of general Yang--Mills theories, giving rise to topologically distinct vacuum states and highly non-perturbative dynamical effects (for an introduction, see \cite{Tong_lecture_notes,lectures-on-instantons,instantons-whats-happening}).




The choice of the symbol $θ$ in the name ``$θ$-term'' reflects the cyclic property of any coefficient $θ$ attached to the Chern--Simons term:
If the Lagrangian density includes such a term
\begin{align}
	\LL_θ[\tsbm F] = \frac{θ}{8\pi^2}\angbr{\tsbm F\wedge\tsbm F}
,\end{align}
then its contribution to the action in the path-integral \eqref{eqn:path-integral} is $θ n$.
Since
\begin{math}
	e^{iθ n} = e^{i(θ + 2\pi)n}
,\end{math}
the coefficient $θ$, henceforth the \emph{$θ$-parameter}, is naturally an angular quantity.
The Chern--Simons term is therefore also referred to as the $θ$-term when included in a Lagrangian.





\begin{figure}[t]
\begin{aside}[Glossary of technical terms]
	\begin{itemize}[leftmargin=0.75em]
		% \setlength{\itemindent}{1em}
		\setlength\itemsep{0.25ex}
		\item \emph{global symmetry}
	--- a symmetry of the Lagrangian that acts uniformly across spacetime.
		\item \emph{local symmetry}
	--- a symmetry of the Lagrangian which acts independently at each point in spacetime. Local symmetries give rise to gauge bosons.
		\item \emph{chiral}
	--- a transformation acting differently on left- and right-handed fermions.
		\item \emph{anomalous symmetry}
	--- a symmetry of the classical Lagrangian, but not of the measure $\cal D[\bm ψ, \tsbm F]$ in the path integral \eqref{eqn:path-integral}, and hence not a symmetry of the associated quantum theory.
		\item \emph{spontaneously broken symmetry}
	---	a symmetry of the Lagrangian which is not exhibited by the ground state solutions.
		\item \emph{Nambu--Goldstone boson}
	--- scalar bosons that arise due to spontaneously broken symmetries. There is exactly one Nambu--Goldstone boson for each generator of the symmetry which is broken.
	\end{itemize}
\end{aside}
\end{figure}




\subsection{Dirac Fermion Fields and the Chiral Anomaly}
\label{sec:fermions-and-the-chiral-anomaly}

An important kind of matter field in the standard model is the \emph{Dirac fermion field} $\bm φ$, which transforms under the spin-$\frac12$ representation of the Lorentz group $\mathrm{SO}^+(1,3)$.
(Dirac fermions describe the quarks and leptons of the standard model.)
The Dirac matrices $\bm γ^μ$ are basis elements of the Clifford algebra $\cal{Cl}_{1,3}(\mathds{C})$ satisfying $\bm γ^{(μ}\bm γ^{ν)} = \eta^{μν}\bm 1$, and are used to write 4-component Dirac fermions in the spin-$\frac12$ representation.
An inner product on fermions $\angbr{\bm ψ, \bm φ} \equiv \bar{\bm ψ}\bm φ \in \mathds R$ \note{or $\mathds C$?} is provided by the Dirac adjoint $\bar{\bm ψ} \coloneqq \bm ψ^\dagger \bm γ^0$, so that Lorentz-invariant quantities may be naturally constructed.
Finally, fermions may be separated into left-handed $\bm φ_+$ and right-handed $\bm φ_-$ components with the projection operators $\bm φ_\pm = \frac12(1 \pm \bm γ^5)$ where $\bm γ^5 = i\bm γ^0\bm γ^1\bm γ^2\bm γ^3$.
In theories which violate parity, left- and right-handed fermions may experience different interactions.
(For instance, left-handed neutrinos interact in the standard model, while right-handed neutrinos are completely inert.)

The simplest fermion equation of motion (namely, the Dirac equation) derives from the \emph{Dirac Lagrangian density}
\begin{align}
	\LL_\text{Dirac} = \bar{\bm φ}\qty(i\bm γ^μ\partial_μ - m)\bm φ \;\vol
,\end{align}
which describes a non-interacting spin-$\frac12$ fermion of mass $m$.
% Consider a theory whose matter field $\bm ψ$ may consist of one or more fermion fields $\bm φ$...
% If a theory possesses a local gauge symmetry $\bm ψ \mapsto g \cdot \bm ψ$ for $g \in G$, then the Dirac Lagrangian may be localised, giving
The Dirac Lagrangian may be localised in the presence of gauge symmetry,
\begin{align}
	\LL_\text{Dirac} = \bar{\bm ψ}\qty(i\bm γ^μ\nabla_μ - m)\bm ψ \;\vol
	\label{eqn:Dirac-Lagrangian}
,\end{align}
where $\ts\nabla\bm ψ \equiv (\nabla_μ\bm ψ) \, \ts\dd x^μ$ is the covariant derivative with respect to the gauge field, $\tsbm A$.
The matter field $\bm ψ = \bm φ_{(1)} \oplus \cdots \oplus \bm φ_{(n)}$ may now more generally be comprised of multiple fermion fields, which may be rotated into one another by the gauge group (an example being \emph{flavour symmetry} in QCD).
The localised Dirac Lagrangian describes fermions and their interactions with the gauge bosons.
The electromagnetic interactions of charged fermionic matter are described by \eqref{eqn:Dirac-Lagrangian} in the case of a $\U(1)$ gauge symmetry, where the gauge field $\tsbm A = \ts A$ is the electromagnetic vector potential.

In quantum theories of fermions, the $θ$-term $\angbr{\tsbm F\wedge\tsbm F}$ also arises in the Lagrangian via the \emph{chiral anomaly}.
The Dirac Lagrangian classically possesses two fundamental $\U(1)$ symmetries which rotate fermion phases \emph{vectorially} and \emph{axially}.
The chiral anomaly is the breaking of the classical axial symmetry upon quantisation due to subtleties of renormalisation and the path integral formulation \eqref{eqn:path-integral}.
% \note{Introduce this smoothly.}
% Such QFTs exhibit two important symmetries relating to the fermion fields, and both give rise to conserved currents.
% However, one of these sy
% This is a symmetry of the classical Lagrangian which is violated in the corresponding quantum theory.

The vectorial symmetry $\U(1)_V$ is the invariance of the Dirac Lagrangian under global Abelian transformations known as \emph{vector fermion rotations}, parametrised by $α$ as
\begin{align}
	\bm ψ &\overset{\U(1)_V}{\mapsto} e^{iα}\bm ψ
% ,&	\eval{\fdv{\bm ψ}{α}}_0 = i \bm ψ
.\end{align}
% Since $\eval{\fdv*{\bm ψ}{α}}_0 = i \bm ψ$, the conserved Noether current density associated to this symmetry is
The conserved Noether current density associated to this symmetry is
\begin{align}
	\ts J_V %&= \pdv{\ts{\cal L}_\text{Dirac}}{\ts\nabla ψ} \, \fdv{\bm ψ}{α}
	&= \bar{\bm ψ}\star\!\tsbm γ\, \bm ψ
,&	\text{i.e.,}\quad
	j_V^{\,μ} &= \bar{\bm ψ}\,\bm γ^μ \bm ψ
,\end{align}
where $\tsbm γ = \bm γ_μ \, \ts\dd x^μ$.
(These 3-form and vector representations are related by $\ts J_V = \star\ts j_V$.)
% \begin{align}
% 	\ts j_V = \bar{\bm ψ}\,\bm γ^μ \bm ψ \, \ts\partial_μ
% \end{align}
The associated continuity equation reads $\ts\dd\ts J_V = 0$ (i.e., $\partial_μj_V^μ = 0$).
Classically, this corresponds to conservation of charge, and in the quantum regime, to conservation of the number of charged particles.

On the other hand, the axial symmetry $\U(1)_A$ is invariance under \emph{axial fermion rotations}, which transform left- $\bm ψ_+$ and right-handed $\bm ψ_-$ fermion components differently:
\begin{align}
	\bm ψ &\overset{\U(1)_A}\mapsto e^{i\bm γ^5 θ}\bm ψ
,&	\text{i.e.,} \quad
	\bm ψ_\pm &\overset{\U(1)_A}\mapsto e^{\pm iθ}\bm ψ_\pm
	\label{eqn:chiral-rotation}
.\end{align}
The associated Noether current density,
\begin{align}
	\ts J_A &= \bar{\bm ψ} \star\!\tsbm γ \, \bm γ_5 \, \bm ψ
,&	\text{i.e.,}\quad
	j_A^μ &= \bar{\bm ψ} \, \bm γ^μ \bm γ_5 \,\bm ψ
	% \ts J_A &= \bar{\bm ψ}\,\bm γ_μ \bm γ_5 \bm ψ \star\ts\dd x^μ
,\end{align}
is conserved classically.
The charge associated to $\ts J_A$ is the number of particles minus antiparticles, named the \emph{baryon number} in QCD.
However, $\U(1)_A$ symmetry is \emph{anomalous}, meaning it does not survive the renormalisation procedure, and consequently, is not an exact symmetry of the quantum theory.\footnote{
	Indeed, baryon number need not conserved in QCD (although violations have not been observed).
	This may provide a possible explanation of \emph{baryogenesis}---the mechanism by which the early universe produced more matter than antimatter \cite[§\,24.6]{ParticleDataGroup-review-2020}.
}
Instead, the continuity equation $\ts\dd\ts J = 0$ fails by the presence of none-other than the $θ$-term,
\begin{align}
	\ts\dd\ts J_A = \angbr{\tsbm F \wedge \tsbm F} = \ts\dd\csform
	\quad\text{\note{coefficient?}}
	\label{eqn:chiral-anomaly}
.\end{align}
% Redefining $\ts J'_A = \ts J_A - \csform$ so that $\ts\dd\ts J'_A = 0$ generates the term $$ 
By Noether's theorem, the Lagrangian which yields the true conserved axial current $\ts J_A' = \ts J_A - \csform$ possesses the $θ$-term 
\begin{math}
	\frac{θ}{8π^2}\angbr{\tsbm F \wedge \tsbm F}
.\end{math}
In other words, the the axial rotation \eqref{eqn:chiral-rotation}, does not leave the Lagrangian invariant; but instead generates the $θ$-term.

\note{What does this mean??
Relevant because Peccei-Quinn mechanism depends on chiral symmetry breaking.
Why don't we just put $\ts K = \angbr{\tsbm F \wedge \tsbm F}$ in the Lagrangian, so that $\ts\dd\ts J_A = 0$?}

\section{QCD and the Strong \CP\ Problem}

Quantum chromodynamics describes the strong interactions between hadronic matter.
Pure $N$-quark QCD is a quantum gauge theory whose matter field (a direct product of $N$ fermionic Dirac fields) describes $N$ quark types, or \emph{flavours}, and whose gauge group $\SU(N)$ gives rise to $N^2 - 1$ gauge fields, or \emph{gluons}.
The QCD sector of the standard model is an extension of pure QCD where the quarks partition into \emph{up type} and \emph{down type}, and again into three \emph{generations}, so that the standard model contains six quark flavours.
{
\newcommand{\quark}[2]{$\raisemath{0.7ex}{\underset{#2}{\mathrm{#1}}}$}
\renewcommand{\arraystretch}{1.5}
\begin{center}
	\begin{tabular}{r|ccc}
	generation & I & I\hspace{-1pt}I & I\hspace{-1pt}I\hspace{-1pt}I \\
	\hline
	up type & \quark{u}{\SI{2.2}{\mega\eV}} & \quark{c}{\SI{1.3}{\giga\eV}} & \quark{t}{\SI{170}{\giga\eV}} \\
	down type & \quark{d}{\SI{4.7}{\mega\eV}} & \quark{s}{\SI{0.1}{\giga\eV}} & \quark{b}{\SI{4.2}{\giga\eV}} \\
	\end{tabular}
\end{center}
}
% Pure QCD can be interpolated with QED or the standard model, in which case 
% Quarks are fermionic Dirac fields which come in pairs of \emph{up type} and \emph{down type} quarks, each transforming under different representations of weak hypercharge $\mathrm{U}(1)$ (but which are identical in pure QCD).
% Pure QCD is agnostic to the number of quark pairs, but when included in the standard model, three copies (each called a \emph{generation}) of quark pairs are used, meaning the standard model contains six quark fields in total . \note{??}
One of the attractive qualities of pure QCD is that, given the gauge group $G = \SU(3)$ and its action on a matter field $\bm ψ$, the theory is almost completely specified.
That is, there is not much freedom for QCD to be slightly different without being a drastically different theory.

The Lagrangian of pure, $N$-quark QCD is a sum of the Dirac Lagrangian and the Yang--Mills Lagrangian,
\begin{align}
	\LL_\text{QCD} &= \bar{\bm ψ}\qty(i\bm γ^μ\nabla_μ - m)\bm ψ \, \vol- \angbr{\tsbm F \wedge\star \tsbm F}
	% \cal L_\text{QCD} &= \sum_q\overbar{\bm ψ^{(q)}}\qty(i\bm γ^μ\nabla_μ - m)\bm ψ^{(q)} - \angbr{\tsbm F \wedge\star \tsbm F}
% \\	&\equiv \sum_{q=1}^N\overbar{ψ}_a^{(q)}\qty(iγ^{μ a}{}_b\nabla_μ - m\delta^a{}_b)ψ^b_{(q)} - \frac14F^a{}_{bμν}F_a{}^{bμν}
\\	\text{i.e.,} \quad 
	\cal L_\text{QCD} &= \sum_{q=1}^N\overbar{ψ}_a^{(q)}\qty(iγ^{μ a}{}_b\nabla_μ - m\delta^a{}_b)ψ^b_{(q)} - \frac14F^a{}_{bμν}F_a{}^{bμν}
\end{align}
where
$\bm ψ \equiv \bm φ^{(1)} \oplus \cdots \oplus \bm φ^{(N)}$
is the matter field separated into its quark components, each of which is a Dirac fermion.
The gauge field is $\SU(N)$, and its action is 
$\ts\nabla\bm ψ \equiv (\nabla_μ\bm ψ) \, \ts\dd x^μ$ is the covariant derivative with respect to the gluon field, $\tsbm A$.


At this point, we see that the issue which is given the name \emph{the strong \CP\ problem} offers itself as a question about the naturalness the QCD Lagrangian:
As encountered in §\,\ref{sec:YM-and-topological}, there is another term which may be included in the Lagrangian with no additional physical assumptions; the topological $θ$-term.
The $θ$-term also arises as a consequence of the chiral anomaly, as seen in §\,\ref{sec:fermions-and-the-chiral-anomaly}.
In this sense, the most general pure QCD Lagrangian is
\begin{align}
	\LL_\text{QCD} = \overbar{\bm ψ}\qty(i\bm γ^μ\nabla_μ - m)\bm ψ \,\vol
	- \angbr{\tsbm F \wedge\star \tsbm F}
	+ \frac{θ}{8\pi^2}\angbr{\tsbm F \wedge \tsbm F}
	\label{eqn:pure-QCD-Lagrangian}
,\end{align}
which includes the \CP-violating $θ$-term $\angbr{\tsbm F\wedge\tsbm F}$.

The inclusion of the $θ$-term is ``natural'' because we lack reason to exclude it on a theoretical basis: it is Lorentz and gauge invariant, etc., just like its \CP-symmetric counterpart, $\angbr{\tsbm F \wedge\star\tsbm F}$.
If there were no \CP-violating interactions in the theory, then $θ$ could be set to zero on the basis of symmetry---however, the weak interaction is explicitly parity-violating.
In fact, the standard model is asymmetric under all combinations of charge conjugation, $C$; parity $P$; and time-reversal $T$ modulo the prevailing combined $CPT$ symmetry. 
Therefore, the $θ$-term's \CP\ violation is not satisfactory reason for its exclusion.
\note{There is also an argument that the $θ$-term arises due to the rich structure of the QCD vacuum.}

Proceeding without the assumption that $θ = 0$, one finds that the strong force violates \CP.
A physical prediction of the standard model possessing a \CP-violating QCD sector is that the neutron is expected to possess an electric dipole moment of approximate magnitude $|d_n| \approx 10^{-18} \,e\,\mathrm{cm}$.
In reality, current measurements \cite{electric_dipole_neutron_2020} of the neutron's electric dipole moment yield a tight upper bound of $|d_n| \lesssim 10^{-26} \,e\,\mathrm{cm}$, which in turn implies a stringent constraint on the $θ$-term, $|θ| \lesssim 10^{-10}$ \cite{ParticleDataGroup-review-2020}.
The strong \CP\ problem is the question, ``why is $θ$ so small?''

