\chapter{Solutions to the Strong CP Problem}

\section{Peccei--Quinn Theory}

Perhaps the most well-known proposed resolution to the strong \CP\ problem is the Peccei--Quinn theory of the \emph{axion}, first proposed in 1977 \cite{PecceiQuin_1977}.
The axion is well-known in cosmology because of the its allure as a dark matter candidate.

Peccei--Quinn theory explains the small value of $\theta$ by its promotion to a dynamical field whose potential is minimised when $\theta = 0$.
In other words, a new $\mathrm{U}(1)$ symmetry is introduced (whose action is to rotate $\theta$) and adjoined to the gauge group of QCD.
The QCD Lagrangian is modified by the addition of a potential with minimum at $\theta = 0$, which is said to \emph{spontaneously break} the Peccei--Quinn $\mathrm{U}(1)$ symmetry.
The gauge field associated to this $\mathrm{U}(1)$ symmetry is named the \emph{axion field}, and the gauge bosons obtained upon quantisation are named \emph{axions}.

\note{
	\begin{enumerate}
		\item Predictions of PC theory
		\item Theoretical limitations
		\item Experimental evidence
	\end{enumerate}
}