\chapter{Solutions to the Strong CP Problem}

At first sight, the strong \CP\ problem may not appear to be a problem at all.
After all, QCD is a theory which possibly---but not necessarily---admits a term which gives rise to \CP-violating interactions in the strong force and consequently to an electric dipole moment of the neutron.
Empirical data shows that the neutron's electric dipole moment (and hence the \CP-violating term) is experimentally prohibited.
From a phenomenological perspective, it is satisfactory to simply leave the \CP-term out of the theory's Lagrangian and end the story there.

However, there is strong reason to regard the inclusion of the \CP-violating term as natural, with its exclusion requiring theoretical justification.
As mentioned in the previous section, the standard model is not \CP\ symmetric, and hence the \CP-violating $\theta$-parameter is expected to be of $\mathcal{O}(1)$.
Similarly, the advent of instantons in QCD demonstrated that the $\theta$-term does not necessarily vanish \cite{lectures-on-instantons,instantons-whats-happening}.

The strong \CP\ problem differers from other fine-tuning problems in the standard model \note{such as?} in that it is of almost no consequence to everyday physics, and therefore has no anthropic solution.
Whereas variations of the cosmological constant, the weak scale or the quark and lepton masses predict a universe very different to our own, variation of the $\theta$-parameter hardly affects nuclear physics at all, because it is suppressed by quark masses.
% from http://scipp.ucsc.edu/~dine/solutions_of_strong_cp.pdf

A trivial way to `resolve' the strong \CP\ problem is to simply require that \CP\ be a symmetry of the strong force.
However, this is tautological and only reinforces the problem of why \CP\ symmetry appears to be preserved in some sectors of the standard model while it is broken in others.



\section{The Massless Quark Solution}

\section{Peccei--Quinn Theory}

Perhaps the most well-known proposed resolution to the strong \CP\ problem is the Peccei--Quinn theory of the \emph{axion}, first proposed in 1977 \cite{PecceiQuinn_1977}.
The axion is well-known in cosmology because of the its allure as a dark matter candidate.

Peccei--Quinn theory explains the small value of $\theta$ by its promotion to a dynamical field whose potential is minimised when $\theta = 0$.
In other words, a new $\mathrm{U}(1)$ symmetry is introduced (whose action is to rotate $\theta$) and is adjoined to the gauge group of QCD.
The QCD Lagrangian is modified by the addition of a potential with minimum at $\theta = 0$, which is said to \emph{spontaneously break} the Peccei--Quinn $\mathrm{U}(1)$ symmetry.
The gauge field associated to this $\mathrm{U}(1)$ symmetry is named the \emph{axion field}, and the gauge bosons obtained upon quantisation are named \emph{axions}.

\note{
	\begin{enumerate}
		\item Predictions of PC theory
		\item Theoretical limitations
		\item Experimental evidence
	\end{enumerate}
}

\section{Something in the modern literature about axions?}