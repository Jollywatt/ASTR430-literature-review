\chapter{Solutions to the Strong CP Problem}

\section{Peccei--Quinn Theory}

Perhaps the most well-known proposed resolution to the strong \CP\ problem is the Peccei--Quinn theory of the \emph{axion}, first proposed in 1977 \cite{PecceiQuin_1977}.

Peccei--Quinn theory explains the small value of $\theta$ by its promotion to a dynamical field whose potential is minimised when $\theta = 0$.
In other words, a new $\mathrm{U}(1)$ symmetry is introduced (whose action is to rotate $\theta$) and adjoined to the gauge group of QCD.
The QCD Lagrangian is modified by the addition of a potential which spontaneously breaks the Peccei--Quinn $\mathrm{U}(1)$ symmetry.
The particle associated with the quantisation of the field is named the axion.