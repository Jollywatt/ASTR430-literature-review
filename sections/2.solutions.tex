\chapter{Solutions to the Strong \CP\ Problem}

% At first sight, the strong \CP\ problem may not appear to be a problem at all.
% After all, QCD is a theory whose Lagrangian possibly---but not necessarily---admits a term $\propto \angbr{\tsbm F \wedge \tsbm F}$ which gives rise to \CP-violating interactions in the strong force, predicting an electric dipole moment of the neutron.
% Empirical data is consistent with the neutron's electric dipole moment (and hence the \CP-violating term) being zero.
% From a phenomenological perspective, it is satisfactory to simply leave the $\bar θ$-term out of the theory's Lagrangian and end the story there.
% Indeed, a tautological way to `resolve' the strong \CP\ problem is to simply require that \CP\ be a symmetry of the strong force.
% However, this only begs the question of why \CP\ symmetry appears to be preserved in some sectors of the standard model while it is broken in others.

% Furthermore, there is a strong sense that the inclusion of the \CP-violating term is ``natural''.
% That is, we lack reason to exclude it on a theoretical basis: it is Lorentz and gauge invariant, etc.; it is a theoretical prediction of the non-trivial vacuum structure of QCD (instantons); and arises via the chiral anomaly.
% From an empirical perspective, if no \CP-violating interactions were observed in Nature, then $\bar θ$ could be justifiably set to zero on the basis of symmetry.
% However, the weak interaction is explicitly parity-violating.\footnote{
% 	In fact, the standard model is asymmetric under all combinations of charge conjugation, $C$; parity $P$; and time-reversal $T$ modulo the prevailing combined $CPT$ symmetry.
% }
% Hence, the fact that the $\bar θ$-term violates \CP\ is not theoretically satisfactory reason for its exclusion.


% % Similarly, the advent of instantons in QCD demonstrated that the $θ$-term does not necessarily vanish \cite{lectures-on-instantons,instantons-whats-happening}.\note{?????}



% The strong \CP\ problem differers from other fine-tuning problems in the standard model in the sense that it is of almost no consequence to everyday physics.
% Variation of the $θ$-parameter hardly affects nuclear physics at all because its effects are suppressed by the quark masses \cite{Dine_2018}.
% On the other hand, variations of the cosmological constant, for example, predict universes drastically different to our own, and similarly for the value of the weak scale, or the quark and lepton masses.
% Such fine-tuning problems at least have anthropic solutions---but the strong \CP\ problem does not.\footnote{
% 	Given a theory linking the presence of dark matter to the smallness of $θ$, an anthropic solution may exist if it turns out that dark matter is necessary for, e.g., galaxy formation (investigated in \cite{Dine_2018}).
% }
% % from http://scipp.ucsc.edu/~dine/solutions_of_strong_cp.pdf
% The strong \CP\ problem is therefore a compelling theoretical indication that the standard model remains incomplete.

\note{Introduce and outline solutions}


\section{The Massless Quark Solution}

The simplest resolution to the strong \CP\ problem is to stipulate that at least one quark was in fact massless.
If this were true, then $\det\mat m$ would vanish, and the parameter $\bar θ = θ + \arg\det\mat m$ would render unphysical.
The \emph{massless quark solution} is the claim that $\bar θ \approx 0$ because the up quark is massless, $m_u = 0$.

% In particular, $\barθ$ becomes singular when one of the quark masses is zero.
% Therefore, the $\barθ$-parameter, and hence the $θ$-term, are entirely unphysical in the case of a massless quark.

At first sight, this economical resolution to the strong \CP\ problem appears to be in contradiction with the experimentally determined masses of the quarks, all of which are non-zero (including the up quark, with mass $\sim \SI{2.2}{\mega\eV}$).
However, it was realised in the mid-1980s that the mass of the up quark has two contributions in the standard model Lagrangian: not only the Yukawa mass $m_u$ (the `bare mass') introduced above, but also a non-perturbative contribution $m_\text{eff}$ from topological effects (i.e., instantons) \cite{ruling-out-massless-uquark_2020}.
Only the bare masses contribute to the value of $\bar θ$ via the quark mass matrix $\cal M$.
Importantly, it was plausible that this secondary source of the up quark's mass could be of order $m_\text{eff} \approx \SI{2.2}{\mega\eV}$, allowing $m_u$ to vanish while still preserving the up quark's overall mass.

The massless up quark hypothesis remained controversial until \emph{lattice gauge theory} had advanced sufficiently to made numerical simulations of non-perturbative effects in QCD possible.
Consensus was reached by 2019 that the instanton contribution $m_\text{eff}$ is not sufficiently large, and hence that the massless up quark hypothesis is false \cite{ruling-out-massless-uquark_2015,aoki2016review,ruling-out-massless-uquark_2020}.
Instead, another mechanism is required to explain the smallness of the $\barθ$-parameter.





% \section{Peccei--Quinn Theory}

% Perhaps the most famous proposed resolution to the strong \CP\ problem is the Peccei--Quinn theory of the \emph{axion}, first proposed in 1977 \cite{PecceiQuinn_1977}.
% The axion is well-known in cosmology because of the its allure as a dark matter candidate.

% Peccei--Quinn theory explains the small value of $θ$ by its promotion to a dynamical field whose potential is minimised when $\barθ = 0$.
% In other words, a new $\U(1)$ symmetry is introduced (whose action is to rotate $θ$) and is adjoined to the gauge group of QCD.
% The QCD Lagrangian is modified by the addition of a potential with minimum at $θ = 0$, which is said to \emph{spontaneously break} the Peccei--Quinn $\U(1)$ symmetry.
% The gauge field associated to this $\U(1)$ symmetry is named the \emph{axion field}, and the gauge bosons obtained upon quantisation are named \emph{axions}.

% \note{
% 	\begin{enumerate}
% 		\item Predictions of PC theory
% 		\item Theoretical limitations
% 		\item Experimental evidence
% 	\end{enumerate}
% }



\section{The Peccei--Quinn Mechanism}


Perhaps the most famous resolution to the strong \CP\ problem is the Peccei--Quinn theory of the \emph{axion}, first proposed in 1977 \cite{PecceiQuinn_1977}.
% In essence, the axion solution involves making the $\bar θ$-parameter to a field in such a way that which is dynamically relaxed to zero.
In essence, the axion solution involves extending the standard model in order to promote the original $\bar θ$-parameter to a field $\bar θ(x) = \bar θ_\text{SM} + a(x)$ in such a way that it is dynamically relaxed to zero.
In doing so, a new massive boson described by the scalar axion field $a(x)$ is necessarily introduced.
There are different inequivalent ways to extend the standard model to realise the axion solution, but all Peccei--Quinn axion models share the same necessary features:
\begin{itemize}
	\item The extended Lagrangian $\LL^\text{\,SM}_{+\text{PQ}}$ possesses an additional global chiral symmetry $\UPQ$.
	The action of this \emph{Peccei--Quinn symmetry} depends on the particular axion model, and is not of central importance.
	The main feature of $\UPQ$ is that it is chiral, so that a $\UPQ$ transformation by $α$ radians anomalously induces a $θ$-term $α\angbr{\tsbm F\wedge\tsbm F}$ in the effective Lagrangian (via the chiral anomaly).

	\item The Peccei--Quinn symmetry $\UPQ$ is \emph{spontaneously broken}, and the single resulting Nambu--Goldstone boson is named the axion field, $a(x)$.
	Being a Nambu--Goldstone boson, the axion transforms as $a(x) \mapsto a(x) + α$ under a $\UPQ$ rotation of $α$ radians.
	The chiral anomaly results in a potential for the axion $a(x)$, which gives rise to axion mass, and whose potential minimum occurs where $a(x) = -\bar θ_\text{SM}$.
\end{itemize}

If the $\UPQ$ symmetry was indeed an exact gauge symmetry, then the strong \CP\ problem is trivially solved, because a $\UPQ$ rotation by $\bar θ$ radians cancels the $\bar θ$-term in the effective Lagrangian, meaning the dynamics of the theory are equivalent to one where $\bar θ = 0$.
However, $\UPQ$ need not be exact: if $\UPQ$ is spontaneously broken, then $\bar θ$ is still driven to zero because it obtains a potential from the chiral anomaly.

\subsubsection{A Toy Axion Model}

To illustrate the Peccei--Quinn mechanism explicitly, consider a minimal toy axion model, which involves the addition of two fields to the standard model: a complex scalar field $Φ$, the \emph{parent field}; and an additional fermion $q$.
The extended Lagrangian takes the form
\begin{align}
	\LL_\text{PQ} = \LL^{\bar θ}_\text{SM} + \angbr{\ts\dd Φ, \ts\dd Φ} + \RR(\bar q_+Φq_-) \vol + \LL_q
,\end{align}
where $\LL^{\bar θ}_\text{SM}$ is the standard model Lagrangian including the $\bar θ$-term, $\angbr{\ts\dd Φ, \ts\dd Φ}$ is a kinetic term, $\RR(\bar q_+Φq_-)$ is a Yukawa coupling term, and $\LL_q$ stands for any other terms involving the new fermion $q$.
The action of $\UPQ$ on these fields is
\begin{align}
	Φ &\mapsto e^{i2α}Φ
,&	q &\mapsto e^{i\bm γ^5α}q
	\quad\text{or}\quad
	\left\{
	\begin{aligned}
	\bar q_+ &\mapsto e^{iα}\bar q_+
\\	q_- &\mapsto e^{iα}q_-
	\end{aligned}
	\right.
	\label{eqn:toy-axion-transformation}
,\end{align}
which indeed leaves $\angbr{\ts\dd Φ, \ts\dd Φ}$ and $\RR(\bar q_+Φq_-)$ invariant.
% Because of the chiral anomaly, the axial rotation of $q$ gives rise to a term $α \angbr{\tsbm F\wedge\tsbm F}$ in the effective Lagrangian.
Since $q$ undergoes an axial rotation, the entire (effective) Lagrangian $\LL_\text{PQ}$ is only invariant with the addition of the term $α \angbr{\tsbm F\wedge\tsbm F}/8π^2$, due to the chiral anomaly.
Therefore, the entire action of $\UPQ$ is to transform $\bar θ \mapsto \bar θ + α$, as well as the fields as in \eqref{eqn:toy-axion-transformation}.

The final component of the model is to make the parent field $Φ$ spontaneously break.
This may be done by adding a ``Mexican hat'' potential to $\LL_\text{PQ}$ of the form
\begin{align}
	V(|Φ|) = λ\qty(|Φ|^2 - f_a^2)^2
,\end{align}
where the parameter $f_a$ is interpreted as the \emph{axion scale}, the energy below which axion dynamics are relevant.\note{???}
Below this energy scale, the parent field $Φ$ relaxes to some minimum of the form
\begin{align}
	Φ = |Φ|e^{i\arg Φ} = f_a e^{ia/f_a}
,\end{align}
where $a$ varies across space.
This field $a(x)$ is the Nambu--Goldstone boson associated with the spontaneous breaking of $\UPQ$ by the parent field $Φ$, and is identified as the axion field.
\begin{align}
	\LL_\text{PQ} = \LL_\text{SM} + \angbr{\ts\dd a, \ts\dd a} + \qty(\bar θ - \frac{a}{f_a})\frac{1}{8π^2}\angbr{\tsbm F\wedge\tsbm F} + \LL_q
,\end{align}


The Yukawa term \note{how/why} can be normalised by a $\UPQ$ rotation by $-a/f_a$
\begin{align}
	\RR(\bar q_+ Φ q_-) = f_a\RR(\bar q_+ e^{ia/f_a} q_-)
	\mapsto f_a\RR(\bar q_+ q_-)
,\end{align}
inducing an axial rotation of $q$ and a corresponding rotation of $θ \mapsto θ - a/f_a$.





\clearpage





The axion is well-known in cosmology because of its allure as a dark matter candidate.
It arises as a Nambu--Goldstone boson of an additional spontaneously broken symmetry, and dynamically solves the strong \CP\ problem via the \emph{Peccei--Quinn mechanism}.
\note{There are a variety of implementations (``models'') of the Peccei--Quinn mechanism.}





The Peccei--Quinn mechanism involves extending the standard model (usually by the addition of fields) so that the entire Lagrangian possesses an additional approximate global chiral $\U(1)$ symmetry, named the Peccei--Quinn symmetry $\UPQ$.
(Different ways of realising $\UPQ$ in an extended standard model lead to different axion models.
In the simplest case, a complex scalar field $φ$ is adjoined to the standard model, whose phase transforms under $\UPQ$ as $φ \mapsto e^{iα}φ$.)
If $\UPQ$ was indeed an exact symmetry, then combined with existing $\U(1)_A$ axial fermion rotations, there would be enough gauge freedom to rotate $\bar θ$ to zero.
Similar to the massless quark solution, this would render the $\bar θ$-parameter unphysical thereby trivially solving the strong \CP\ problem.
However, $\UPQ$ cannot be an exact symmetry because it is subject to the chiral anomaly (see §\,\ref{sec:fermions-and-the-chiral-anomaly}).
What Peccei and Quinn showed was that even if the $\UPQ$ symmetry is not exact, then $\bar θ$ would be dynamically driven to zero---provided that $\UPQ$ was also spontaneously broken.
The chiral anomaly and spontaneous breakage together translate into an effective potential for $\bar θ$, whose minimum is at zero.


Because $\UPQ$ is chiral, like $\U(1)_A$, it is also broken by the chiral anomaly.
Consequently, the Noether current $\ts J_\text{PQ}$ associated to the $\UPQ$ symmetry is not conserved, but instead obeys
\begin{align}
	\ts\dd\ts J_\text{PQ} &= ξ \angbr{\tsbm F\wedge\tsbm F}
,&	\text{i.e.,} \quad \partial_\mu j_\text{PQ}^\mu &= \fracξ4\angbrAd{\bm F_{μν}, \bm F_{\rho\sigma}}\epsilon^{μν\rho\sigma}
,\end{align}
where $ξ$ is a parameter particular to the axion model of interest.
Simultaneously, $\UPQ$ is spontaneously broken.
In the simple case of a scalar field $φ$, spontaneous $\UPQ$-breakdown means that the expectation value of $φ$ in a vacuum state is not zero, but some complex number of fixed magnitude, $φ = |φ_0|e^{iα}$, which is not invariant under $\UPQ$.
The phase $α$ is an effective $\U(1)$ degree of freedom which, in the quantum theory, is a spin-$0$ particle.
This particle---the axion---is the \emph{pseudo-Nambu--Goldstone boson} associated to the spontaneously broken $\UPQ$ symmetry.


As with the axial rotation symmetry, the anomalous Peccei--Quinn symmetry gives rise to an extra term in the Lagrangian.
\note{I don't understand how.}
\begin{align}
	\LL_\text{anomaly} = \frac{a}{f_a}ξ\angbr{\tsbm F\wedge\tsbm F}
.\end{align}


% The non-conservation of $\ts J_\text{PQ}$ implies the presence 
% Because of the chiral anomaly, however, L_eff must also have a term in which the axion field couples directly to the gluon density GG̃, so as to guarantee that J PQ has the correct QCD chiral anomaly

The \emph{chiral solution} to the strong \CP\ problem postulates that the standard model Lagrangian possesses another global chiral symmetry, in addition to the symmetry \eqref{eqn:quark-mass-phase-transform}.
In their original paper \cite{PecceiQuinn_1977}, Peccei and Quinn showed that the existence of such a symmetry would resolve the strong \CP\ problem, even if it was spontaneously broken (in which case, another boson would arise---the \emph{axion}).
This Peccei--Quinn symmetry $\UPQ$ is specified by the action of $\U(1)$ on quark fields by a chiral rotation $\bm ψ_{(q)} \mapsto e^{i\bm γ^5 a}\bm ψ_{(q)}$, with $a \in [0, 2π)$, enabling the phases of all quark flavours to be rotated uniformly in a chiral manner.

If $\UPQ$ were in fact an exact symmetry of the standard model Lagrangian, then, used in conjunction with the chiral symmetry \eqref{eqn:quark-mass-phase-transform}, $\bar θ$ could be set to zero by a simple choice of gauge, rendering it unphysical. \note{I do not actually understand how this works.}
However, $\UPQ$ is not an exact symmetry of the standard model.
Nevertheless, Peccei and Quinn showed that if $\UPQ$ was a symmetry that was spontaneously broken, then the parameter $\bar θ$ would be dynamically driven to zero \cite{Peccei_1996}.

