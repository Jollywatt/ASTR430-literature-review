\chapter{Solutions to the Strong \CP\ Problem}

At first sight, the strong \CP\ problem may not appear to be a problem at all.
After all, QCD is a theory whose Lagrangian possibly---but not necessarily---admits a term $\propto \angbr{\tsbm F \wedge \tsbm F}$ which gives rise to \CP-violating interactions in the strong force, predicting an electric dipole moment of the neutron.
Empirical data is consistent with the neutron's electric dipole moment (and hence the \CP-violating term) being non-existent.
From a phenomenological perspective, it is satisfactory to simply leave the $θ$-term out of the theory's Lagrangian and end the story there.
Indeed, a tautological way to `resolve' the strong \CP\ problem is to simply require that \CP\ be a symmetry of the strong force.
However, this only begs the question of why \CP\ symmetry appears to be preserved in some sectors of the standard model while it is broken in others.

Furthermore, there is a strong sense that the inclusion of the \CP-violating term is natural while its exclusion requires theoretical justification.
As mentioned in the previous section, the standard model is not \CP\ symmetric, and hence the \CP-violating $θ$-parameter is expected to be of $\mathcal{O}(1)$: why it is so small is a fine-tuning problem.
% Similarly, the advent of instantons in QCD demonstrated that the $θ$-term does not necessarily vanish \cite{lectures-on-instantons,instantons-whats-happening}.\note{?????}

The strong \CP\ problem differers from other fine-tuning problems in the standard model in the sense that it is of almost no consequence to everyday physics.
Variation of the $θ$-parameter hardly affects nuclear physics at all because its effects are suppressed by the quark masses \cite{Dine_2018}.
On the other hand, variations of the cosmological constant predict universes drastically different to our own, and similarly for the value of the weak scale, or the quark and lepton masses.
Such fine-tuning problems at least have anthropic solutions---but the strong \CP\ problem does not.\footnote{
	Given a theory linking the presence of dark matter to the smallness of $θ$, an anthropic solution may exist if it turns out that dark matter is necessary for, e.g., galaxy formation (investigated in \cite{Dine_2018}).
}
% from http://scipp.ucsc.edu/~dine/solutions_of_strong_cp.pdf
The strong \CP\ is therefore a strong theoretical indication that the standard model remains incomplete.


\section{The Massless Quark Solution}


In particular, $\barθ$ becomes singular when one of the quark masses is zero.
Therefore, the $\barθ$-parameter, and hence the $θ$-term, are entirely unphysical in the case of a massless quark.
The \emph{massless quark solution} to the strong \CP\ problem is the claim that $\barθ = 0$ because at least one quark is in fact massless.

At first sight, this economical resolution to the strong \CP\ problem appears to be in contradiction with the experimentally determined masses of the quarks, all of which are non-zero (including the up quark, whose mass is estimated to be $\sim 2\,\mathrm{MeV}$).
However, it was realised in the mid-1980s that the mass of the up quark has two contributions in the standard model Lagrangian: not only the Yukawa mass $m_u$ (the `bare mass') introduced above, but also a non-perturbative contribution $m_\text{eff}$ from topological effects (i.e., instantons) \cite{ruling-out-massless-uquark_2020}.
Only the bare masses contribute to the value of $\bar θ$ via the quark mass matrix $\cal M$.
Importantly, it was plausible that this secondary source of the up quark's mass is of order $m_\text{eff} \approx 2\,\mathrm{MeV}$, allowing $m_u$ to vanish by another mechanism while still preserving the up quark's overall mass.

The massless up quark hypothesis remained controversial until \emph{lattice gauge theory} had advanced sufficiently to made numerical simulations of non-perturbative effects in QCD possible.
Consensus was reached by 2019 that the instanton contribution $m_\text{eff}$ is not sufficiently large, and hence that the massless up quark hypothesis is false \cite{ruling-out-massless-uquark_2015,aoki2016review,ruling-out-massless-uquark_2020}.
Instead, another mechanism is required to explain the smallness of the $\barθ$-parameter and solve the strong \CP\ problem.





% \section{Peccei--Quinn Theory}

% Perhaps the most famous proposed resolution to the strong \CP\ problem is the Peccei--Quinn theory of the \emph{axion}, first proposed in 1977 \cite{PecceiQuinn_1977}.
% The axion is well-known in cosmology because of the its allure as a dark matter candidate.

% Peccei--Quinn theory explains the small value of $θ$ by its promotion to a dynamical field whose potential is minimised when $\barθ = 0$.
% In other words, a new $\U(1)$ symmetry is introduced (whose action is to rotate $θ$) and is adjoined to the gauge group of QCD.
% The QCD Lagrangian is modified by the addition of a potential with minimum at $θ = 0$, which is said to \emph{spontaneously break} the Peccei--Quinn $\U(1)$ symmetry.
% The gauge field associated to this $\U(1)$ symmetry is named the \emph{axion field}, and the gauge bosons obtained upon quantisation are named \emph{axions}.

% \note{
% 	\begin{enumerate}
% 		\item Predictions of PC theory
% 		\item Theoretical limitations
% 		\item Experimental evidence
% 	\end{enumerate}
% }



\section{The Peccei--Quinn Mechanism}


Perhaps the most famous proposed resolution to the strong \CP\ problem is the Peccei--Quinn theory of the \emph{axion}, first proposed in 1977 \cite{PecceiQuinn_1977}.
The axion is well-known in cosmology because of the its allure as a dark matter candidate.
It arises as a Nambu--Goldstone boson of an additional spontaneously broken symmetry, and dynamically solves the strong \CP\ problem via the \emph{Peccei--Quinn mechanism}.
\note{There are a variety of implementations (``models'') of the Peccei--Quinn mechanism.}

The Peccei--Quinn mechanism involves extending the standard model (usually by the addition of fields) so that the entire Lagrangian possesses an additional approximate global chiral $\U(1)$ symmetry, named the Peccei--Quinn symmetry $\UPQ$.
(Different ways of realising $\UPQ$ in an extended standard model lead to different axion models.
In the simplest case, a complex scalar field $φ$ is adjoined to the standard model, whose phase transforms under $\UPQ$ as $φ \mapsto e^{iα}φ$.)
If $\UPQ$ was indeed an exact symmetry, then combined with existing $\U(1)_A$ axial fermion rotations, there would be enough gauge freedom to rotate $\bar θ$ to zero.
Similar to the massless quark solution, this would render the $\bar θ$-parameter unphysical thereby trivially solving the strong \CP\ problem.
However, $\UPQ$ cannot be an exact symmetry because it is subject to the chiral anomaly (see §\,\ref{sec:fermions-and-the-chiral-anomaly}).
What Peccei and Quinn showed was that even if the $\UPQ$ symmetry is not exact, then $\bar θ$ would be dynamically driven to zero---provided that $\UPQ$ was also spontaneously broken.
The chiral anomaly and spontaneous breakage together translate into an effective potential for $\bar θ$, whose minimum is at zero.


Because $\UPQ$ is chiral, like $\U(1)_A$, it is also broken by the chiral anomaly.
Consequently, the Noether current $\ts J_\text{PQ}$ associated to the $\UPQ$ symmetry is not conserved, but instead obeys
\begin{align}
	\ts\dd\ts J_\text{PQ} &= ξ \angbr{\tsbm F\wedge\tsbm F}
,&	\text{i.e.,} \quad \partial_\mu j_\text{PQ}^\mu &= \fracξ4\angbrAd{\bm F_{μν}, \bm F_{\rho\sigma}}\epsilon^{μν\rho\sigma}
,\end{align}
where $ξ$ is a parameter particular to the axion model of interest.
Simultaneously, $\UPQ$ is spontaneously broken.
In the simple case of a scalar field $φ$, spontaneous $\UPQ$-breakdown means that the expectation value of $φ$ in a vacuum state is not zero, but some complex number of fixed magnitude, $φ = |φ_0|e^{iα}$, which is not invariant under $\UPQ$.
The phase $α$ is an effective $\U(1)$ degree of freedom which, in the quantum theory, is a spin-$0$ particle.
This particle---the axion---is the \emph{pseudo-Nambu--Goldstone boson} associated to the spontaneously broken $\UPQ$ symmetry.


As with the axial rotation symmetry, the anomalous Peccei--Quinn symmetry gives rise to an extra term in the Lagrangian.
\note{I don't understand how.}
\begin{align}
	\LL_\text{anomaly} = \frac{a}{f_a}ξ\angbr{\tsbm F\wedge\tsbm F}
.\end{align}

% Instead of coupling to 

% The non-conservation of $\ts J_\text{PQ}$ implies the presence 
% Because of the chiral anomaly, however, L_eff must also have a term in which the axion field couples
% μ
% directly to the gluon density GG̃, so as to guarantee that J PQ
% has the correct QCD chiral anomaly

% The \emph{chiral solution} to the strong \CP\ problem postulates that the standard model Lagrangian possesses another global chiral symmetry, in addition to the symmetry \eqref{eqn:quark-mass-phase-transform}.
% In their original paper \cite{PecceiQuinn_1977}, Peccei and Quinn showed that the existence of such a symmetry would resolve the strong \CP\ problem, even if it was spontaneously broken (in which case, another boson would arise---the \emph{axion}).
% This Peccei--Quinn symmetry $\UPQ$ is specified by the action of $\U(1)$ on quark fields by a chiral rotation $\bm ψ_{(q)} \mapsto e^{i\bm γ^5 a}\bm ψ_{(q)}$, with $a \in [0, 2π)$, enabling the phases of all quark flavours to be rotated uniformly in a chiral manner.

% If $\UPQ$ were in fact an exact symmetry of the standard model Lagrangian, then, used in conjunction with the chiral symmetry \eqref{eqn:quark-mass-phase-transform}, $\bar θ$ could be set to zero by a simple choice of gauge, rendering it unphysical. \note{I do not actually understand how this works.}
% However, $\UPQ$ is not an exact symmetry of the standard model.
% Nevertheless, Peccei and Quinn showed that if $\UPQ$ was a symmetry that was spontaneously broken, then the parameter $\bar θ$ would be dynamically driven to zero \cite{Peccei_1996}.

