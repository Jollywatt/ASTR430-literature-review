\documentclass[a4paper,12pt,twoside]{report}
\usepackage[margin=3cm,bindingoffset=-1.2cm]{geometry}

%% PACKAGES %%

% Bibliography
\usepackage[
	style=phys,
	biblabel=brackets,
	maxbibnames=4, % write `et al.' for many authors
	% sorting=none,
]{biblatex}
	\addbibresource{citations.bib}
	\DeclareFieldFormat{titlecase}{#1}

% Graphics packages
\usepackage{xcolor}

% Math packages
\usepackage{amsmath}
\usepackage{amssymb} % defines \mathfrak
\usepackage[
	math-style=ISO, % italic Greek
	warnings-off={mathtools-colon,mathtools-overbracket}]
{unicode-math}
\usepackage{physics}
\usepackage{accents}
\usepackage{mathrsfs} % defines \mathscr
\usepackage{dsfont}
% \usepackage{siunitx}
% \usepackage{eucal}
% \usepackage{dutchcal}
\usepackage{mathtools}
	\mathtoolsset{showonlyrefs} % autonum seems far inferior


% Layout packages
\usepackage[inline]{enumitem}
\usepackage[margin=\parindent,font=footnotesize]{caption}

% Misc. packages
\usepackage[ddmmyy]{datetime}


\usepackage{hyperref}
	\hypersetup{
		colorlinks,
		linkcolor={red!50!black},
		citecolor={blue!50!black},
		urlcolor={blue!80!black}
	}

\listfiles

%% LAYOUT %%

% David's preferred line spacing
\setlength{\parskip}{4pt}

% Omit page numbering (until chapter 1)
\pagenumbering{gobble}

% Set chapter style
\usepackage{titlesec}
\titleformat{\chapter}[hang]{\normalfont\huge\bfseries}
	{\thechapter}{1em}{}
% "1 Title" vs "Chapter 1\nTitle"

% hyphenation rules for funky words
\hyphenation{
	Schwarz-schild
}

% Environment for outlined 'aside boxes'
\usepackage[framemethod=default]{mdframed}
\newenvironment{aside}[1][Note]{ 
\vspace{-1.6ex}
\begin{mdframed}[
	frametitle={\colorbox{white}{\normalfont\emph{#1}}},
	frametitleaboveskip=-1.6ex,
	frametitlealignment=\raggedright,
	font=\small,
	linewidth=.3pt,
	leftmargin=1cm,
	rightmargin=1cm,]
\vspace{-1.6ex}
}{\end{mdframed}}


%% COMMANDS %%

% \scalemath - scale math symbols easily
\newcommand\scalemath[2]{
	\scalebox{#1}{\mbox{\ensuremath{\displaystyle #2}}}
}

% \raisemath - raise and lower math symbols, obeying sub/superscript contexts
\makeatletter
\newcommand{\raisemath}[1]{\mathpalette{\raisem@th{#1}}}
\newcommand{\raisem@th}[3]{\raisebox{#1}{$#2#3$}}
\makeatother

% \currentfontsize - refers to original font size
\makeatletter
\newcommand{\currentfontsize}{
	\fontsize{\f@size}{\f@baselineskip}\selectfont
}
\makeatother


%% SYMBOLS & SHORTCUTS %%

\newcommand{\CP}{{CP}}

\renewcommand{\cal}{\mathcal}
\renewcommand{\frak}{\mathfrak}

\newcommand{\bm}{\symbf}
\newcommand{\ts}[1]{\underaccent{\sim}{#1}} % David's tensor 'twiddle'
\newcommand{\tsbm}[1]{\ts{\bm{#1}}}

\newcommand{\secs}{\symrm{\Gamma}}
\newcommand{\forms}{\symrm{\Omega}}

\newcommand{\T}{\mathrm{T}}
\newcommand{\TT}{\mathds{T}}
\newcommand{\LL}{\ts{\mathscr L}}
\newcommand{\M}{\mathcal{M}}

\newcommand{\ang}[1]{\left\langle{#1}\right\rangle}
\newcommand{\angg}[1]{\ang{#1}_{\raisemath{2pt}{\mathrm{Ad}}}}

\newcommand{\fibrebundle}[4][]{#2 \rightarrowtail #3 \overset{#1}{\twoheadrightarrow} #4}

% \newcommand{\csform}{\phi_{\text{CS}}^{(3)}}
\newcommand{\csform}{\ts\omega_3}
\newcommand{\vol}{\ts{\mathrm{vol}}}

\newenvironment{baligned}
	{\left\{\begin{aligned}}
	{\end{aligned}\right.}


%% DRAFT & FINAL MODE %%

\newif\iffinal
\newcommand{\note}[1]{\iffinal\else\textcolor{gray}{<<#1>>}\fi}
\newcommand{\urgent}[1]{\iffinal\else\textcolor{red}{\{#1\}}\fi}

% uncomment to render as final, hiding notes to self
% \finaltrue


\begin{document}

\newgeometry{margin=2cm}

\begin{titlepage}


\begin{center} \linespread{0.9}
\Large School of Physical and Chemical Sciences, University of Canterbury, Christchurch, New Zealand
\end{center}

\vspace{3cm}

\begin{center}
\huge\bf
An Introduction to the Strong CP Problem, its Solutions, and Axions (ó\_ò,)
\end{center}

% \vspace{1ex}

\begin{center} \linespread{1.5}
	\Large
	Joseph A.\ Wilson
\\	Supervisor Assoc.\ Prof.\ Jenni Adams
\end{center}
% \centerline{\Large
% }

\iftrue
\vspace{3cm}
\centerline{\note{\textsc{Compiled on {\today}, {\currenttime}}}}
\fi

\vfill

\begin{center} \linespread{1.5}
\includegraphics[width=45mm]{resources/UC.png} \\
\vspace{1ex}
\Large \textsc{Astr430 Literature Review} \\
% Supervisor Assoc.\ Prof.\ Jenni Adams
\end{center}
\vspace{4pt}
% \centerline{\it ``You've got to learn to do monkey tricks!''}

\vspace{2cm}

\end{titlepage}

\restoregeometry


\begin{abstract}
	The standard model of particle physics suffers from the strong \CP problem---the fine tuning of the QCD $\bar θ$-parameter to the experimental value $|\bar θ| \lesssim \num{1e-10}$.
	This review discusses the theoretical background of QCD and of the strong \CP problem, and describes its most famous resolution: the Peccei--Quinn axion.
	The axion has wide-ranging implications for cosmology and astrophysics, and up-to-date constraints of its properties from various laboratory and cosmological experiments are reviewed.
\end{abstract}

\tableofcontents
\newpage
\pagenumbering{arabic}

\setcounter{chapter}{-1}

\chapter{Introduction}

\begin{itemize}
	\item overall
	\item Axions as a dark matter candidate
\end{itemize}

\include{sections/1.background}

\chapter{Solutions to the Strong CP Problem}

At first sight, the strong \CP\ problem may not appear to be a problem at all.
There is a theory, which possibly---but not necessarily---admits a term which gives rise to \CP-violating interactions in the strong force and consequently to an electric dipole moment of the neutron.
Upon measurement, the neutron's electric dipole moment, and the \CP-violating term is ruled out by experiment.
From a phenomenological perspective, it is satisfactory to simply end the story here.

The strong \CP\ problem differers from other fine-tuning problems in the standard model \note{such as} in that it is of almost no consequence to everyday physics, and thus has no anthropic solution.
Whereas variations of the cosmological constant, the weak scale or the quark and lepton masses lead to universe very different to our own, variation of the $\theta$-parameter hardly affects nuclear physics because it is suppressed by quark masses.
% from http://scipp.ucsc.edu/~dine/solutions_of_strong_cp.pdf

A trivial way to `resolve' the strong \CP\ problem is to simply require that \CP\ be a symmetry of the strong force.
However, this is tautological and only reinforces the problem of why \CP\ symmetry appears to be preserved in some sectors of the standard model while it is broken in others.

\section{The Massless Quark Solution}

\section{Peccei--Quinn Theory}

Perhaps the most well-known proposed resolution to the strong \CP\ problem is the Peccei--Quinn theory of the \emph{axion}, first proposed in 1977 \cite{PecceiQuinn_1977}.
The axion is well-known in cosmology because of the its allure as a dark matter candidate.

Peccei--Quinn theory explains the small value of $\theta$ by its promotion to a dynamical field whose potential is minimised when $\theta = 0$.
In other words, a new $\mathrm{U}(1)$ symmetry is introduced (whose action is to rotate $\theta$) and adjoined to the gauge group of QCD.
The QCD Lagrangian is modified by the addition of a potential with minimum at $\theta = 0$, which is said to \emph{spontaneously break} the Peccei--Quinn $\mathrm{U}(1)$ symmetry.
The gauge field associated to this $\mathrm{U}(1)$ symmetry is named the \emph{axion field}, and the gauge bosons obtained upon quantisation are named \emph{axions}.

\note{
	\begin{enumerate}
		\item Predictions of PC theory
		\item Theoretical limitations
		\item Experimental evidence
	\end{enumerate}
}

\section{Something in the modern literature about axions?}

\chapter{Axions in Cosmology}

\begin{figure}[t!]
\begin{aside}[Terminology from cosmology]
\begin{itemize}[leftmargin=0.75em]\setlength\itemsep{0.25ex}
	\item \emph{thermalisation}
---	the process of a particle species reaching thermal equilibrium (i.e., uniform energy and abundance) over cosmological scales, mitigated by self-interactions or processes involving other matter species which diffuse energy.
	\item \emph{freeze-out}
---	the point beyond which the rates of thermalising processes become negligible due to a sufficiently large rate of cosmic expansion, resulting in persistent non-equilibrium distributions of a particle species.
Since the universe cools as it expands, freeze-out may be viewed as the change of phase of a species from a gas to a cooler condensate.
\end{itemize}
\end{aside}
\end{figure}

With the advent of precision cosmology, particle physicists can use the entire universe as a laboratory for ever more sensitive experiments.

Basicaly:
\begin{itemize}
	\item Model universe as homogeneous and isotropic (FLRW)
	\item State of universe is defined by the abundances of each species.
\end{itemize}

\begin{align}
	Γ_{a\to γγ} \approx \SI{1.1e-24}{\per\second}\qty(\frac{m_a}{\si{\eV}})^5
\end{align}



\section*{Acknowledgements}
\addcontentsline{toc}{chapter}{Acknowledgements}

I would like to thank my supervisor Jenni Adams for taking me up on what was meant to be a one-semester project, and for letting me wander largely in my own direction and at my own pace.
Thank you for your input and good spirit.

\begingroup
\let\clearpage\relax
\addcontentsline{toc}{chapter}{Bibliography}
\printbibliography
\endgroup



\end{document}

