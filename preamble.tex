\documentclass[a4paper,12pt,twoside]{report}
\usepackage[margin=3cm,bindingoffset=-1.2cm]{geometry}

%% PACKAGES %%

% Typography
\usepackage{libertine}

% Bibliography
\usepackage[
	style=phys,
	biblabel=brackets,
	maxbibnames=4, % write `et al.' for many authors
	% sorting=none,
]{biblatex}
	\addbibresource{citations.bib}
	\DeclareFieldFormat{titlecase}{#1}

% Graphics packages
\usepackage{xcolor}

% Math packages
\usepackage{amsmath}
\usepackage{amssymb} % defines \mathfrak
\usepackage[
	math-style=ISO, % italic Greek
	warnings-off={mathtools-colon,mathtools-overbracket}]
{unicode-math}
\usepackage{physics}
\usepackage{accents}
\usepackage{mathrsfs} % defines \mathscr
\usepackage{dsfont}
\usepackage{siunitx}
% \usepackage{eucal}
% \usepackage{dutchcal}
\usepackage{mathtools}
	\mathtoolsset{showonlyrefs} % autonum seems far inferior


% Layout packages
\usepackage[inline]{enumitem}
\usepackage[margin=\parindent,font=footnotesize]{caption}

% Misc. packages
\usepackage[ddmmyy]{datetime}


\usepackage{hyperref}
	\hypersetup{
		colorlinks,
		linkcolor={red!50!black},
		citecolor={blue!50!black},
		urlcolor={blue!80!black}
	}

\listfiles

%% LAYOUT %%

% David's preferred line spacing
\setlength{\parskip}{4pt}

% Omit page numbering (until chapter 1)
\pagenumbering{gobble}

% Set chapter style
\usepackage{titlesec}
\titleformat{\chapter}[hang]{\normalfont\huge\bfseries}
	{\thechapter}{1em}{}
% "1 Title" vs "Chapter 1\nTitle"

% hyphenation rules for funky words
\hyphenation{
	Schwarz-schild
}

% Environment for outlined 'aside boxes'
\usepackage[framemethod=default]{mdframed}
\newenvironment{aside}[1][Note]{ 
\vspace{-1.6ex}
\begin{mdframed}[
	frametitle={\colorbox{white}{\normalfont\emph{#1}}},
	frametitleaboveskip=-1.6ex,
	frametitlealignment=\raggedright,
	font=\small,
	linewidth=.3pt,
	leftmargin=1cm,
	rightmargin=1cm,]
\vspace{-1.6ex}
}{\end{mdframed}}


%% COMMANDS %%

% \scalemath - scale math symbols easily
\newcommand\scalemath[2]{
	\scalebox{#1}{\mbox{\ensuremath{\displaystyle #2}}}
}

% \raisemath - raise and lower math symbols, obeying sub/superscript contexts
\makeatletter
\newcommand{\raisemath}[1]{\mathpalette{\raisem@th{#1}}}
\newcommand{\raisem@th}[3]{\raisebox{#1}{$#2#3$}}
\makeatother

% \currentfontsize - refers to original font size
\makeatletter
\newcommand{\currentfontsize}{
	\fontsize{\f@size}{\f@baselineskip}\selectfont
}
\makeatother


%% SYMBOLS & SHORTCUTS %%

\newcommand{\CP}{{$CP$}}

\renewcommand{\cal}{\mathcal}
\renewcommand{\frak}{\mathfrak}

\newcommand{\bm}{\symbf}
\newcommand{\ts}[1]{\underaccent{\sim}{#1}} % David's tensor 'twiddle'
\newcommand{\tsbm}[1]{\ts{\bm{#1}}}

\newcommand{\secs}{\symrm{Γ}}
\newcommand{\forms}{\symrm{\Omega}}

\renewcommand{\U}{\mathrm{U}} % I hope this is ok
\newcommand{\SU}{\mathrm{SU}}
\newcommand{\UPQ}{\U(1)_\text{PQ}}

\newcommand{\T}{\mathrm{T}}
\newcommand{\TT}{\mathds{T}}
\newcommand{\LL}{\ts{\mathscr L}}
\newcommand{\M}{\mathcal{M}}

\newcommand{\angbr}[1]{\left\langle{#1}\right\rangle}
\newcommand{\angbrAd}[1]{\angbr{#1}_{\raisemath{2pt}{\mathrm{Ad}}}}

\newcommand{\fibrebundle}[4][]{#2 \rightarrowtail #3 \overset{#1}{\twoheadrightarrow} #4}

% \newcommand{\csform}{\phi_{\text{CS}}^{(3)}}
\newcommand{\csform}{\ts\omega_3}
\newcommand{\vol}{\ts{\mathrm{vol}}}

\newenvironment{baligned}
	{\left\{\begin{aligned}}
	{\end{aligned}\right.}


%% DRAFT & FINAL MODE %%

\newif\iffinal
\newcommand{\note}[1]{\iffinal\else\textcolor{gray}{<<#1>>}\fi}
\newcommand{\urgent}[1]{\iffinal\else\textcolor{red}{\{#1\}}\fi}

% uncomment to render as final, hiding notes to self
% \finaltrue
